

\begin{frame}
	\frametitle{Проблема беспорядочных потоков}

	Уравнение течение вязкой жидкости:
	$$
	\frac {\partial {\vec {v}}}{\partial t}=-({\vec {v}}\cdot \nabla ){\vec {v}}+\nu \Delta {\vec {v}}-{\frac {1}{\rho }}\nabla p+{\vec {f}}
	$$

	Для возникновения турбулентного течения число Рейнольда должно превысить некоторое $\text{R}_{\text{крит.}}$

	$$\text{R} = \frac{\rho v L}{\eta}$$
\end{frame}

\begin{frame}
	\frametitle{Развитие теории возникновения турбулентности}

	До возникновения же турбулентоного течения последовательное повышение $\text{R}$ приводит к движению потоков по следующим сценариям:

	\phantom{42}\\
	\textbf{стационарное $\rightarrow$ периодичное $\rightarrow$ квазипериодичное}

	\phantom{42}\\
	Разберём их подробнее.
\end{frame}