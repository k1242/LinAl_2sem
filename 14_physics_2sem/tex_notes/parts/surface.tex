\section{Поверхностное натяжение}

\noindent
\textbf{Формула Лапласа}:
\begin{equation}
p_{in}-p_{out} = \sigma \left(\frac{1}{r_1} + \frac{1}{r_2}\right)
\end{equation}

\noindent
\textbf{Термодинамика поверхности}\\
Работа сил поверхностного натяжения при уменьшении площади на $d \Pi$:
\begin{equation}
\delta A = - \sigma d \Pi
\end{equation}
Формулировка первого начала для квазистатических процессов:
$$T d S = d U - \sigma d \Pi$$
Внутренняя энергия поверхности
$$d F = - T d S + \sigma d \Pi$$
Так как $F = U - T S$:
$$U = F - T \ptdr{F}{T}{\Pi}$$
Представим свободную энергию в виде $F = \sigma \Pi$. Тогда:
\begin{equation}
U = \left(\sigma - T \frac{d \sigma}{ d T} \right) \cdot \Pi
\end{equation}