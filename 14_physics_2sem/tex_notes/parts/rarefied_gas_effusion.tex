\subsection*{Эффузия разреженного газа}

\begin{to_def}[Эффузия]
    Эффузия -- медленное истечение газа через малые отверстия.
\end{to_def}

Истечение характеризуется \textit{числом Кнудсена} $\Kn \lambda/L$, где $\lambda$ -- длина свободного пробега, $L$ -- характерный размер системы. При $\Kn \gg 1$ столкновениями между молкекулами в окрестности отверстия можно пренебречь. Такое течение называется \textit{свободным молекулярным течением}.

\phantom{42}

\noindent
\textbf{Эффект Кнудсена:}
\begin{equation}
    jS = \frac{1}{4}Sn \overline{v} = Sn \sqrt{\frac{kT}{2 \pi m}}
\end{equation}

\begin{proof}[$\triangle$]
Сравняем потоки через отверстие. В некотором приближении:
$
\dfrac{P_1}{\sqrt{T_1}} = \dfrac{P_2}{\sqrt{T_2}}.
$
\end{proof}


\noindent
\textbf{Блуждание тяжелой молекулы:}
\begin{equation}
    \overline{(M \triangle v)^2} \sim 
    N \cdot \overline{(m v_0)^2}, 
    \hspace{1cm}
    \lambda \sim \frac{\overline{v}}{\overline{v_0}} \frac{1}{n \sigma }, \hspace{1cm} \lambda' \sim N \lambda
\end{equation}


\noindent
\textbf{Движение макроскопической частицы:}
\begin{equation}
    \vc{u} = b \vc{F_{\T{внеш}}}, \hspace{1cm}
    \boxed{
    M \frac{d\vc{u}}{dt} = - \frac{\vc{u}}{b} + \tilde{\vc{F}}(t)
    }
\end{equation}


\noindent
\textbf{Закон Пуазейля:}
\begin{equation}
    Q = \frac{\pi R^4}{5 \eta l} (P_1 - P_2)
\end{equation}