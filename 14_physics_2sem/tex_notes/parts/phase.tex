\section{Фазовые переходы}
Химический потенциал $\mu = \ptdr{U}{N}{S,\, V}$. Тогда
$$d U = T d S - P d V + \mu d N$$

Интенсивные величины не зависят от числа частиц в системе, а экстенсивные пропорциональны этому числу, поэтому можно записать:
$$ \Phi = \Phi (P, T, N) = N \cdot f(P,T) $$
То есть $\mu = \frac{\Phi}{N}$. Таким образом, химических потенциал есть термодинамический потенциал в расчете на одну частицу.

Поскольку $d \Phi = - S d T + V d P + \mu d N $ и $\Phi = \mu N $:
$$d \mu = - s d T + v d P,$$
где $s$ и $v$ -  энтропия и обьем в расчете на одну частицу.

Рассмотрим двухфазную систему, помещенную в жесткую адиабатическую оболочку.Совершим бесконечно малый процесс, в котором фазы находятся в тепловом и механическом равновесии:
\begin{align*}
    d U_1 = T d S_1 - P d V_1 + \mu_1 d N_1 \\
    d U_2 = T d S_2 - P d V_2 + \mu_2 d N_2
\end{align*}
Понятно что $d U_1 = -d U_2$ , $d V_1 = -d V_2$ и $d N_1 = -d N_2.$
Тогда:
$$T d S = (\mu_1 - \mu_2) d N$$
Энтропия в состоянии равновесия имеет максимум, значит 
$$\mu_1 = \mu_2$$
Из того, что $d \mu_1 = d \mu_2$ можно получить уравнение \textbf{Клапейрона - Клаузиуса} :
\begin{equation} \boxed{
\frac{d P}{d T}= \frac{q}{ T (v_1 - v_2)}
}, \end{equation}
где $ q = s_1 - s_2$. Кстати тут важно, что энтропия именно удельная, иначе бы это не работало 

\phantom{42}

\noindent
\textbf{Зависимоcть давления насыщенного пара от температуры}\\
Из уравнения Клапейрона-Клаузиуса можно получить (газ идеальный, $ v_1 \gg v_2$):
$$\frac{d P}{d T}= \frac{q}{ k T^2} P,$$ тогда
\begin{equation}
P = P_0 \exp\left(\frac{q}{k T_0} - \frac{q}{k T}\right).
\end{equation}

\phantom{42}

\noindent
\textbf{Давление насыщенного пара над искривленной поверхностью}

Пренебрегая зависимостью давления насыщенного пара от температуры:
$$P = P_0 - \frac{ g h}{v_1}$$
Найдем  h. Для разницы давлений под поверхностью и над поверхностью жидкости:
$$\Delta P = \rho_2 g h =  = \frac{ g h}{v_2} = \sigma K, $$
где $K$ - кривизна поверхности, $v_1$ и $\rho_1$ - удельный обьем и плотность пара $v_2$ и $\rho_2$ --  удельный обьем и плотность жидкости.
В итоге:
\begin{equation}
    P = P_0 + \frac{v_2}{v_1 - v_2}\sigma K
\end{equation}
Если нельзя пренебречь изменением плотности пара с высотой, то есть $ P = P_0 \exp \left(- \dfrac {\mu g h }{R T }\right)$:
\begin{equation}
   \ln\frac {P}{P_0} = \frac{\mu v_2}{R T} \cdot (P - P_0 - \sigma K)
\end{equation}
Если  $\dfrac{\mu v_2}{R T} \cdot |P - P_0| \ll 1 $ и $|\sigma K| \gg |P - P_0| $:
\begin{equation}
\boxed{
P = P_0 \exp \left(\dfrac {\mu v_2}{RT} \sigma K\right)
}.
\end{equation}
