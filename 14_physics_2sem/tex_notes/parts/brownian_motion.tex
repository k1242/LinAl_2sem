\noindent
\textbf{Закон Эйнштейна-Смолуховского}
\begin{equation}
\label{EI_S}
    \begin{split}
        1\text{D:}\\
        2\text{D:}\\
        3\text{D:}
    \end{split} \hspace{1cm}
    \begin{split}
        \overline{\vc{r}^2} = r_0^2 + 2 Dt \\
        \overline{\vc{r}^2} = r_0^2 + 4 Dt \\
        \overline{\vc{r}^2} = r_0^2 + 6 Dt 
    \end{split}
\end{equation}

\begin{proof}[$\triangle$]
\com{
1) Сила торможения $\vc{F}_{\T{тр}} = \vc{v}/B$, где $B$ -- \textit{подвижность частицы}. В частном случае шариков: $B = 1/(6 \pi R \eta )$. }

\com{
2) Флуктуационная сила $\vc{X}$ со стороны молекул среды: $\overline{\vc{X}} = 0$.
}

\com{
Тогда уравнение\footnote{
    Уравнение, учитывающее детерминированные и случайные силы называется \textit{уравнение Ланжевена}. 
} движения частицы:
$$
m \frac{d^2 r}{dt^2} = \vc{X} - \frac{1}{B} \frac{d \vc{r}}{dt}.
$$
Умножая почленно на $\vc{r}$ и усредняя, получим\footnote{
    Учтём, что 
    $\vc{r} \dfrac{d^2 \vc{r}}{dt^2} = 
    \dfrac{1}{2} \dfrac{d^2 (\vc{r})^2}{dt^2} - 
    \lr{\dfrac{d \vc{r}}{dt}}^2$.
}:
$$
\frac{m}{2} \frac{d^2}{dt^2} \overline{\vc{r}^2} +
\frac{1}{2B} \frac{d}{dt} \overline{\vc{r}^2} =
\overline{m \vc{v}^2} + 
\overline{\vc{rX}}.
$$
Далее учтём, что $\overline{\vc{rX}} = 0$, $\overline{mv^2}/2 = 3kT/2$ (другой множитель для другой размерности!). Таким образом:
$$
\lr{\frac{m}{2} \frac{d^2}{dt^2} + \frac{1}{2B}\frac{d}{dt}} \overline{\vc{r}^2} = 3kT,
$$
решение которого 
$$
\overline{\vc{r}^2} = r_0^2 + 6kTBt + C \exp \lr{-\frac{t}{mB}}.
$$
При достаточно больших $t \gg mB$, получим \eqref{EI_S}.
}
\end{proof}

\noindent
\textbf{Связь подвижности и коэффициента диффузии (формула Эйнштейна):}
\begin{equation}
\label{eq_E}
    D = kTB
\end{equation}
\begin{proof}[$\triangle$]
\com{
Из определения подвижности $\vc{u} = B \cdot \vc{F}$. Под действитем $\vc{F}$ возник поток: $j_x^(F) = n u_x = n B F_x$. Пусть $\vc{F}$ -- пот. сила, тогда по распределению Больцмана
$$
n(\vc{r}) = n_0 \exp \lr{-\frac{U(\vc{r})}{kT}}.
$$
Тогда $x$-компонента диффузионного потока равна (по закону Фика)
$$
j_x^{(D)} = - D \ptdv{n}{x} = \frac{D}{kT} n_0 \exp \lr{-\frac{U}{kT}} \ptdv{U}{x} = 
- n \frac{D}{kT} F_x.
$$
В равновесии $j_x^{(D)}+j_x^{(F)} = 0$. Получаем \textit{формулу Эйнштейна} \eqref{eq_E}. 
}
\end{proof}

\noindent
\textbf{Броуновское движение, как случайные блуждания:}
\begin{equation}
    \boxed{\dots}
\end{equation}

\noindent
\textbf{Скорость диффузии и теплопроводности:}
\begin{equation}
    L \sim \sqrt{Dt}, \hspace{1cm} L \sim \sqrt{at}
\end{equation}