\section{Уравнение состояния}

\begin{table}[h]
\caption{Коэффициенты}
    \centering
    \begin{tabular}{c|c|c}
    \toprule
    %  name 1 &  name 2  \\
    % \midrule
        $\alpha = \dfrac{1}{V_0} \ptdr{V}{T}{P} $ &
        $\gamma  = \dfrac{1}{P_0} \ptdr{P}{T}{V}$ & 
        $\beta_T =   -V          \ptdr{P}{V}{T} $ \\
    \bottomrule
    \end{tabular}
    \label{tab1}
\end{table}

\subsection{Тепловое расширение тел}
В окрестности минимума потенциальной энергии разложим ее в ряд 
$$ V(r) = V(d) + \frac{1}{2}B x^2 + \frac{1}{3}C x^3 \dots , $$
где $x = r - d$.\\

При $C=0$, $\overline x = 0$.
Если раскладывать дальше:
$$\overline x^2 \approx \frac{k T}{B}, \;  \overline x = - \frac{C}{B^2}k T.$$
Для общего удлинения цепочки:
$$ \overline r = d(1 + \alpha T).$$
В итоге
$$\alpha = \frac{\overline r - d}{d}\frac{1}{T} = -\frac{C}{B^2 d} k$$





\subsection{Внутренняя энергия идеального газа}
\noindent
\textbf{STM}. Теплоемкость $C_V$ являеся функцией только температуры.
\begin{proof}
    Рассмотрим \textit{адиабатический} процесс Джоуля-Томпсона. Газ прошел от $P_1$ до $P_2$:
    \begin{align*}
        \delta A + dU  = 0, & \; \; \; \; \; A = P_2 V_2 - P_1 V_1 \\
        \text{(на опыте) } PV = \nu R T &\Rightarrow T_1 = T_2 \\
        \text{таким образом: } & \Rightarrow U = U(T)
    \end{align*}
\end{proof}

\noindent
\textbf{EQ}. Уравнение политропы:
\begin{equation}
    P V^{ \gamma } = const, \; \text{где } \gamma = \frac{c - c_P}{c - c_V}
\end{equation}



