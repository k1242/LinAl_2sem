
\section{\fbox{Статистика классических систем}}

\sbs{Теорема Лиувилля}
Пусть рассматриваемая макроскопическая система имеет $s$ степеней свободы. То есть положение точек системы описывается $s$ координатами, тогда её состояние в данный момент определяется $s$ координатами $q_i$ и соответствующих им $s$ скоростей $p_i$ (в статистике принято использовать импульсы).

Мысленно выделим из замкнутой макро-системы весьма малую, но всё ещё макроскопическую \textbf{подсистему}, которая уже не будет замкнутой системой. Решим задачу об эволюции подсистемы статистически:

Обозначим $\Delta p \Delta q$ некоторый малый участок фазового объёма пространства подсистемы (с соответствующими $p_i, q_i$ и $\Delta p_i, \Delta q_i$). В течении достаточно большого времени $T$ фазовая траектория много раз пройдёт через всякий такой $\Delta p \Delta q$. 
Пусть $\Delta t$ -- та часть $T$, в течение которой траектория находилась в данном $\Delta p \Delta q$. Введём вероятность наблюдения системы в данном участке: $\omega = \lim_{T \rightarrow \infty} \frac{\Delta t}{T}$. 

А для бесконечно малых:	$\d \omega = \rho(p_1,\ldots,p_s,q_1,\ldots,q_s) \d p \d q$, где $\rho$ - функция всех координат и импульсов системы, \textbf{функция статистического распределения}. Также можно вычислить статистически \textbf{среднее значение} любой величины $f(p,q)$: $\bar{f} = \int f(p,q) \rho (p,q) \d p \d q$. 
Усреднение с помощью $\rho$ освобождает от необходимости следить за зависимостью от времени величины $f(p,q)$, однако, если бы мы знали $f = f(t)$, то $\bar{f} = \lim_{T \rightarrow \infty} \frac{1}{T} \int_0^T f(t) \d t$.

В выделенных макроскопических подсистемах с окружающими взаимодействуют лишь частицы, находящиеся на границе, однако таких частиц мало, по сравнению с частицами в объёме всей подсистемы, тогда можно считать такие подсистемы \textbf{квазизамкнутыми} - замкнутыми в течении небольшого промежутка времени. Тот факт, что различные подсистемы слабо взаимодействуют друг с другом, приводит к тому, что можно считать их \textbf{независимыми}, то есть $\rho_{1 2}= \rho_1 \rho_2$, а также $\overline{f_1 f_2} = \bar{f_1}\bar{f_2}$

Рассмотрим колебания величины $f$ с течением времени. Введём величину характеризующую в среднем ширину интервала этого изменения: $\Delta f = f - \bar{f}$, или $\langle (\Delta f)^2\rangle = \overline{f^2} - \bar{f}^2$ -- корень из которой называется \textbf{среднеквадратичной флуктуацией} величины $f$. И чем меньше отношение $\langle (\Delta f)^2\rangle/\bar{f}$ (относительная флуктуация), тем более вероятно нахождения $f$ в ничтожном отклонении от своего среднего значения.


И так, мы наблюдаем в течении длительного времени некоторую подсистему. 
Разделим этом промежуток времени на очень большое количество одинаковых маленьких интервалов, разделенных моментами времени $t_1, t_2,\ldots$, в каждый момент подсистема изобразится точкой фазового пространства $A_1, A_2, \ldots$. 
В пределе совокупность таких точек будет пропорциональна $\rho(p,q)$. 
Чтобы не утруждаться рассмотрением одной подсистемы в разные моменты времени можно ввести \textbf{статистический ансамбль}  -- большое число совершенно одинаковым образом устроенных подсистем находящихся в момент времени $t$ в состояниях $A_1,A_2,\ldots$ В каждый последующий момент времени все точки будут распределены в пространстве согласно функции распределения $\rho(p,q)$.

Формально, передвижения фазовых точек можно рассматривать как стационарное течение "газа" в $2 s$-мерном фазовом пространстве. Применим к нему известное уравнение непрерывности, выражающее собой неизменность числа фазовых точек.
\begin{equation*}
	\frac{\partial p}{\partial t} + \text{div} (\rho \vc{v}) = 0. \hspace{1cm} \text{Для стационарного течения: div}(\rho \vc{v}) = 0
\end{equation*}
Обобщение последнего выражения на случай $2 s$-мерного пространства (где $x_i$ - координаты $q$ и импульсы $p$):
\begin{equation}
 	\sum\limits_{ i=1 }^{ 2 s } \frac{\partial}{\partial x_i} (\rho v_i)= 0 
 	\leadsto
 	\sum\limits_{ i=1 }^{ s } \left[ \frac{\partial}{\partial q_i} (\rho \dot{q}_i) + \frac{\partial}{\partial p_i}(\rho \dot{p}_i)\right] = 0
 	\leadsto
 	\sum\limits_{ i=1 }^{ s } \left[\dot{q}_i \frac{\partial \rho}{\partial q_i} + \dot{p}_i \frac{\partial \rho}{\partial p_i} \right] +
 	\rho \sum\limits_{ i=1 }^{ s } \left[\frac{\partial \dot{q}_i}{\partial q_i} + \frac{\partial \dot{p}_i}{\partial p_i}\right] = 0.
 	\label{3.1}
 \end{equation} 

 Зная уравнения механики в форме Гамильтона: $\dot{q}_i = \frac{\partial H}{\partial p_i}$, $\dot{p}_i = - \frac{\partial H}{\partial q_i}$ $\leadsto$ $\frac{\partial \dot{q}_i}{\partial q_i} = - \frac{\partial \dot{p}_i}{\partial p_i}$. Поэтому второй член уравнения \ref{3.1} тождественно обращается в нуль. Первый же член -- ни что иное, как полная производная от функции распределения по времени. Таким образом:
 \begin{equation*}
 	\frac{\d \rho}{\d t} = \sum\limits_{ i=1 }^{ s } \left( \frac{\partial \rho}{\partial q_i} \dot{q}_i  +  \frac{\partial \rho}{\partial p_i} \dot{p}_i \right) = 0.
 \end{equation*}
 Таким образом мы получили \textbf{теорему Лиувилля}: функция распределения постоянна вдоль фазовых траекторий подсистемы. Полученный результат получен для квазизамкнутых подсистем.


 Из теоремы Лиувилля следует, что $\rho$ должна выражаться через комбинации $p$ и $q$, которые при движении подсистемы как замкнутой остаются постоянной. Это -- механические инварианты или интегралы движения.
 Можно сказать, что $\rho$ сама есть интеграл движения.

 Сузим число интегралов движения от которых может зависеть функция распределения, зная, что $\rho_{1 2} = \rho_1 \rho_2 \leadsto \ln \rho_{1 2} = \ln \rho_1 + \ln \rho_2$. Следовательно логарифм функции распределения должен быть аддитивным интегралом движения.
 Таких независимых интеграл в механики существует всего семь: энергия, три компоненты вектора импульса и три компоненты момента импульса.
 Единственная аддитивная комбинация этих величин -- линейная, тогда для $a$-ой подсистемы:
 \begin{equation}
 	\ln \rho_a = \alpha_a + \beta E_a (p,q) + \vc{\gamma} \vc{P_a} (p,q) + \vc{\delta} \vc{M_a} (p,q)
 \end{equation}
 Коэффициент $\alpha_a$ -- просто нормировочная постоянная.
 Постоянные же $\beta, \vc{\gamma}, P\vc{\delta}$ -- всего семь независимых значений, которые могут быть определены по семи же постоянным значениям аддитивных интегралов движения замкнутой подсистемы.

 Таким образом приходим к выводу: значения аддитивных интегралов движения полностью определяют статистические свойства замкнутой системы. Эти интегралы заменяют собой невообразимое множество данных, которое бы требовалось при механическом подходе.

Для того чтобы в дальнейшем исключить из рассмотрения момент и импульс, будем представлять себе систему заключенную в твердый ящик и пользоваться системой координат в которой ящик покоится. В таких условиях момент и импульс вообще не будут интегралами движения, что позволяет перейти к формуле: 
\begin{equation}
	\ln \rho_a - \alpha_a + \beta E_a(p,q) \leadsto \text{микроканоническое распределение: } \rho = \const \cdot \delta(E - E_0)
\end{equation}

Иногда, не как выше, интересно рассматривать эволюцию систему на сравнимых с временем релаксации системы промежутках времени, для больших систем такое возможно, так как наряду с полным статистическим равновесием всей замкнутой системы существуют, так называемые, \textbf{неполные равновесия}.
Время релаксации растёт с увеличением размеров системы, в силу этого отдельные малые части системы сами по себе приходят в равновесное состояние системы значительно быстрее, чем происходит установление равновесие между различными малыми частями. 
Это значит, что каждая малая часть системы описывается своей функцией распределения. В таком случае говорят, что системы находится в \textbf{неполном равновесии}. С течением времени неполное равновесие постепенно переходит в полное, причем параметры $\beta,\vc{\gamma},\vc{\delta}$ для каждой маленькой части, медленно изменяясь, со временем в конце концов становятся одинаковыми вдоль всей замкнутой системы.

Наличие неполных равновесий позволяет ввести понятие о \textbf{макроскопических состояниях} системы.

 \sbs{Распределение Гиббса}
 Поскольку статистические распределения подсистем должны быть по самому определению статистического равновесия стационарными, то мы, прежде всего, заключаем, что матрицы $\omega_{m n}$ всей подсистем диагональны.
 Это утверждение связано с пренебрежением взаимодействия систем друг с другом, точнее можно сказать, что недиагональные элементы $\omega_{n m}$ стремятся к нулю по мере уменьшения роли взаимодействия, то есть по мере увеличения числа частиц в подсистемах. 
 Тогда имеем формулу среднего значения какой-либо величины $\bar{f} = \sum w_n f_{n  n}$.

Далее найдём логарифм функции распределения для квазизамкнутой подсистемы a: $\ln \omega_n^{(a)} = \alpha^{(a)} + \beta E_n^a$.
Для математической формулировки этого "квантового микроканонического распределения" надо применить применить следующее.

Имея в виду "почти непрерывность" энергетического спектры макроскопических тел, введём понятие о числе квантовых состояний замкнутой системы, "приходящийся" на определенный малый интервал значений её энергии, $\d \Gamma$.

Если рассматривать замкнутую систему с квазизамкнутыми подсистемами, то число $\d \Gamma = \prod_a \d \Gamma_a$. Теперь мы можем сформулировать микроканоническое распределение, написав для вероятности $\d \omega$ нахождения системы в каком-либо из $\d \Gamma $ состояний:
\begin{equation}
	\d \omega = \const \cdot \delta(E -E_0) \prod_a \d \Gamma_a.
\end{equation}

Нашей цель -- нахождение вероятности нахождения системы в некотором определенном квантовом состоянии с энергией $E_n$, то есть в макроскопически описанном состоянии.
Пусть $\Delta \Gamma'$ -- статистический вес макроскопического состояния среды, обозначим как $\Delta E'$ интервал значений энергии среды, соответствующий интервалу квантовых состояний. Имеем:
\begin{equation}
		\omega_n = \const \cdot \delta(E + E' - E^{(0)})\d \Gamma \d \Gamma' \leadsto
		\omega_n = \const \cdot \int \delta(E_n + E' - E^{(0)}) \d \Gamma',
\end{equation}
где, $E, \d \Gamma$ и $E', \d \Gamma'$ относятся соотвественно к телу и среде, $E^{(0)}$ заданное значение замкнутой энергии системы; положим $\d \Gamma = 1$, $E = E_n$ и проинтегрировав по $\d \Gamma'$.

Пусть $\Gamma'(E')$ -- полное число квантовых состояний среды с энергией не большей $E'$. 
Поскольку подынтегральное выражение зависит только от $E'$, можно перейти к интегрированию по $\d E'$: $\d \Gamma' = \frac{\d \Gamma' (E')}{\d E'}\d E'$. Зная определение энтропии: $\frac{\d \Gamma'}{\d E'} = \frac{\exp(S'(E'))}{\Delta E'}$, где $S'(E')$ -- энтропия среды от её энергии.

Таким образом:
\begin{equation}
	\omega_n = \const \cdot \int \frac{e^{S'}}{\Delta E'} \delta (E' + E_n - E^{(0)})\d E' 
	\overset{E' \to E^{(0)} - E_n}{\leadsto}
	\omega_n = \const \cdot \left((\frac{e^{S'}}{\Delta E'})\right).
\end{equation}
Учтём теперь, что вследствие малости тела его энергия $E_n$ мала по сравнению с $E^{(0)}$, $\Delta E'$ относительно осень мало меняется при незначительном изменении $E'$,
поэтому в ней можно просто положить $E' = E^{(0)}$, не зависящую от $E_n$ постоянную. 
В экспоненциальном же множителе $e^{S'}$ надо разложить энтропию по степеням $E_n$ сохранив линейный член: $S'(E^{(0)}-E_n) = S'(E^{(0)}) - E_n \frac{\d S' (E^{(0)})}{\d E^{(0)}}$.	
Производная от энтропии по энергии $S'$ есть не что иное, как $1/T$, где T -- температура системы (температуры тела и среды считаются одинаковыми из-за равновесия).

Таким образом, получаем окончательно распределение Гиббса:
\begin{equation}
	\boxed{\omega_n = A \exp \left(- \frac{E_n}{T}\right)} \hspace{1cm}
	\bar{f} = \sum\limits_{ n }^{  } \omega_n f_{n n } = \frac{\sum\limits_{ n }^{  } f_{n n} e^{-E_n/T}}{\sum\limits_{ n }^{  } e^{-E_n/T}}.
\end{equation}