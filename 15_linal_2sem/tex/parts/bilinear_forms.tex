\section{Билинейные формы}

\noindent
\socrat. Матрица билинейной формы:
\begin{equation}
     f(\vc{x}, \vc{y}) = x\T B y
\end{equation} 

\noindent
\socrat. Матрица билинейной формы, полярной к квадратичной $q(\vc{x})$:
\begin{equation}
     f(\vc{x}, \vc{y}) = \frac{1}{2} \lr{q(\vc{x} + \vc{y}) - q(\vc{x}) - q(\vc{y})}
\end{equation} 

\begin{to_thr}
    Для  $\forall B\colon B = B\T$ $\exists$ канонический базис.
\end{to_thr}

\noindent
$\boxed{\Rightarrow}$. Для  $\forall B\colon B = B\T$ $\exists$ $A \colon A\T B A$ -- диагональная матрица. 

\noindent
$\boxed{\Rightarrow}$. Всякая $q$ на $V(\mathbb{R})$ приводится к \textit{нормальному} виду:
\begin{equation}
\label{q_norm}
    q(x) = x_1^2 + \dots + x_s^2 - x_{s+1}^2 - \dots - x_r^2.
\end{equation}

\begin{to_thr}[закон инерции]
    Пусть $q$ -- квадр. форма на $V_n(\mathbb{R})$. Тогда $r$ и $s$ из \eqref{q_norm} $\equiv f(q)$.
\end{to_thr}


\begin{to_thr}[метод Якоби]
    \underline{Пусть} $q$ -- квадратичная форма на $V$: все главные миноры отличны от 0. \underline{Тогда} существует базис \vc{e}:  $q(\vc{x})$ принимает канонический вид:
    \begin{equation}
        q(\vc{x}) = \frac{\Delta_0}{\Delta_1} (x_1')^2 + \frac{\Delta_1}{\Delta_2} (x_2')^2 + \dots + \frac{\Delta_{n-1}}{\Delta_n} (x_n')^2
    \end{equation}
\end{to_thr}

\begin{to_thr}[критерий Сильвестра]
    Квадратичная форма $q$ на $V_n(\mathbb{R})$  положительно определена $\Leftrightarrow$ все главные миноры её матрицы $(b_{ij})$ положительно определены.
\end{to_thr}

\begin{to_thr}
    $\forall$ $B: B\T = - B$ размера $2m \times 2m$ конгруэнтна матрице:
    \begin{equation}
        J = \begin{bmatrix}
            0   & -1    &       &   &   \\
            1   & 0     &       &   &   \\
                &       &\dots  &   &   \\
                &       &       &0  &-1 \\
                &       &       &1  &0  
        \end{bmatrix},
    \end{equation}
т.е. найдётся невырожденная матрица $A$: $A\T B A = J$.
\end{to_thr}

