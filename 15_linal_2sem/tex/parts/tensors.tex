\section{Тензоры}
\subsection{Начала тензорного исчисления}
\subsubsection{Понятие о тензорах}
\begin{to_def}[Понятие тензора]
    Пусть $\F$ -- поле, $V(\F)$ - векторное пространство, $V^*$ -- сопряженное к $V$, $p$ и $q$ -- целые числа $\geqslant 0$. Всякое $(p+q)$-линейное отображение 
    \begin{equation}
        f \colon V^p \times (V^*)^q \to \F
    \end{equation}
    называется \textbf{тензором на $V$ типа $(p, q)$} и валентности (или ранга) $p+q$.
\end{to_def}


%%%%%%%%%%%%%%%%%%%%%%%%%%%%%%%%%%%%%%%%%%%%%%%%%%%%%%%%%%%%%%%%%%%%%%%%
\subsubsection{Произведение тензоров}

\begin{to_def}
    Пусть $f \colon V_1 \times \dots \times V_r \to \F$, $g \colon W_1 \times \dots \times W_s \to \F$.
    Под \textit{тензорным произведением} $f$ и $g$ понимают отображение
    \begin{equation}
        f \otimes g \colon V_1 \times \dots \times V_r \times W_1 \times \dots \times W_s \to \F,
    \end{equation}
    определенное формулой
    \begin{equation}
        (f \otimes g)(\vc{v}_1, \dots, \vc{v}_r; \vc{w}_1, \dots, \vc{w}_s) = f(\vc{v}_1, \dots, \vc{v}_r) \cdot g(\vc{w}_1, \dots, \vc{w}_s)
    \end{equation}
\end{to_def}

\noindent
Некоторые свойства тензорного произведения:
\begin{figure}[ht!]
\begin{minipage}{0.5\textwidth}
\center
        \begin{tabular}{c|lll}
            \toprule
            $p/q$   &0          &1                          & 2    \\
            \midrule
            0       & $\const$  & $f$        & $b(\vc{x}, \vc{y})$       \\
            1       & $\vc{x}$  &  $L \in \mathcal{L}(V)$   &       \\  
            2       & $b^*(f, g)$     &&       \\
            \bottomrule
        \end{tabular}
\end{minipage}
\begin{minipage}[b]{0.3\textwidth}
        \begin{tabular}{rll}
        1) & $\otimes \colon \mathbb{T}^{p}_{q} \times \mathbb{T}^{p'}_{q'} \to \mathbb{T}^{p+p'}_{q+q'}$; &\\
        2) & ассоциативность     & \checkmark    \\
            & дистрибутивность  & \checkmark \\
            & \cancel{коммутативность}    & \xmark       \\
        \end{tabular}
\end{minipage}
\end{figure}

\begin{minipage}{0.3\textwidth}
    Базис в $V$ и $V^*$ выбирается:
\end{minipage}
\hfill
\begin{minipage}{0.7\textwidth}
   \begin{equation}
    \begin{aligned}
        (\vc{e}_i, e^j) = \delta_i^j = 
            \left\{
                \begin{aligned}
                    0, \text{ если }  i \neq j, \\
                    1,  \text{ если }  i = j,   \\
                \end{aligned} 
            \right. \\
        f(\vc{x}) = (f, \vc{x}) = \sum\nolimits_i \alpha^i  \beta_i = \alpha^i  \beta_i.
    \end{aligned}
    \end{equation} 
\end{minipage}

\subsubsection{Координаты тензора}

\begin{to_def}[Компоненты тензора]
    Значения тензора обозначаются в виде:
    \begin{equation}
            T_{i_1, \dots, i_p}^{j_1, \dots, j_p} := T(\vc{e}_{i_1}, \dots, \vc{e}_{i_p}, e^{j_1}, \dots, e^{j_q}).
    \end{equation}  
    Числа  $\numt$ называются \textbf{координатами} тензора $T$ в базисе $(\vc{e}_1, \dots, \vc{e}_n)$
\end{to_def}

\begin{to_thr}
    Тензоры типа $(p, q)$ на $V$ составляют $\mathbb{T}_p^q (V)$ размерности $n^{p+q}$ с базисными векторами
    \begin{equation}
        e^{i_1} \otimes \dots \otimes  e^{i_p} \otimes \vc{e}_{j_1} \otimes \dots \otimes \vc{e}_{j_q},
    \end{equation}
    При том $\exists ! T$ с координатами $\numt$.
\end{to_thr}


\begin{proof}[$\triangle$]
Достаточно построить \textit{разложимый} тензор. Далее воспользуемся равенством:
$$
    (e^{i_{1}} \otimes \ldots e^{i_p} \otimes \vc{e}_{j_{1}} \otimes \ldots \otimes \vc{e}_{j_q}) (\vc{e}_{i_{1}'},  \ldots,  \vc{e}_{i_p'}, e^{j_1'}, \ldots,  e^{j_q'}) =
     \delta_{i_1'}^{i_{1}} \ldots \delta_{i_p'}^{i_{p}}
     \delta_{j_1'}^{j_{1}} \ldots \delta_{j_q'}^{j_{q}}.
$$
Построим тензор
$$
    T_{1} = \sum_{i, j} T_{i_1 \ldots i_p}^{j_1 \ldots  j_q} 
    (e^{i_{1}} \otimes \ldots e^{i_p} \otimes \vc{e}_{j_{1}} \otimes \ldots \otimes \vc{e}_{j_q}),
$$
просто линейную комбинацию с некоторой индексацией. Теперь получим
\begin{equation*}
    T_1(\vc{e}_{i_1}, \dots, \vc{e}_{i_p}, e^{j_1}, \dots, e^{j_p}) = \numt,
\end{equation*}
и воспользуемся тем, что тензор $T$ полностью определяется своими координатами. Почему? 

В силу полилинейности для произвольных векторов
$$
    \vc{x}_{1} = \sum\nolimits_{i_{1}} \xi^{i_{1}} \vc{e}_{i_{1}}, \; \;
    \ldots, \; \;
    \vc{x}_p = \sum\nolimits_{i_p} \rho^{i_p} \vc{e}_{i_p}
$$
и линейных форм
$$
    u^1 = \sum\nolimits_{j_{1}} \sigma_{j_{1}} e^{j_{1}}, \; \;
    \ldots, \; \;
    u^p = \sum\nolimits_{j_p} \tau_{j_q} e^{j_q}
$$
имеем   
$$
    T(\vc{x}_{1}, \ldots,  \vc{x}_p, u^1, \ldots,  u^p) = \sum\nolimits_{i, \, j}
    T_{i_1, \dots, i_p}^{j_1, \dots, j_q} \xi^{i_{1}} \ldots \rho^{i_p} \sigma_{j_{1}} \ldots \tau_{j_q}.
$$


Далее остается показать, что разложимые тензоры, отвечающие различным наборам индексов линейно независимы. Это следует из правила вычисления их значений. Пусть они ЛЗ
$$
    \sum \lambda_{i_1, \dots, i_p}^{j_1, \dots, j_p} \xi^{i_{1}} 
    e^{i_{1}} \otimes \ldots e^{i_p} \otimes \vc{e}_{j_{1}} \otimes \ldots \otimes \vc{e}_{j_q} = 0,
$$
где $\lambda_{i_1, \dots, i_p}^{j_1, \dots, j_p} \in \mathbb{F}$. Аналогично с $T_1$ можем подставить элемент базиса, свести к работе с символами Кронекера. 

\end{proof}


\subsubsection{Переход к другому базису}

\begin{to_thr}
    При переходе от дуальных базисов $(\vc{e}_i)$, $(e^i)$ пространств $V$ и $V^*$ к новым дуальным базисам тех же пространств:
    \begin{equation}
        \vc{e}_k' = a^i_k \vc{e}_i, \hspace{1cm} e^{'k} = b^k_i e^i, \text{ где } (a_{ij})^{-1}) = (b_{ij}), 
    \end{equation}
    координаты тензора $T$ преобразуются по формулам
    \begin{equation}
        \numt = \sum_{i', j'} 
        b_{i_1', \dots, i_p'}^{i_1, \dots, i_p}
        \cdot 
        T'\phantom{|}_{i_1, \dots, i_p}^{j_1, \dots, j_p} 
        \cdot
        a_{j_1' \dots j_p'}^{j_1 \dots j_p}
    \end{equation}
\end{to_thr}

% \begin{proof}[$\triangle$]
%     \koz \hspace{1cm} \koz \hspace{1cm} \koz \hspace{1cm} \koz \hspace{1cm} \koz \hspace{1cm} (в принципе, всё понятно) \hspace{1cm} \koz \hspace{1cm} \koz \hspace{1cm} \koz
% \end{proof}


\subsubsection{Тензорное произведение пространств}
\begin{to_thr} 
    Пусть $V, W$ --- векторные пространства над полем $\mathbb{F}$ . Тогда существует векторное пространство $T$ над $\mathbb{F}$ и билинейное отображение $b \colon V \times W \to T$, удовлетворяющее условиям:

             \begin{tabular}{c|rl}
                (T1) &
                если $\vc{v}_{1}, \ldots, \vc{v}_k \in V$ ЛНеЗ и $\vc{w}_{1}, \ldots, \vc{w}_k \in W,$  & 
                то $\sum_{i=1}^k b(\vc{v}_i, \vc{w}_i) = 0$ $\Longrightarrow$ $\vc{w}_{1}=\ldots =\vc{w}_k=0$;\\
                (Т2) &
                если $\vc{w}_{1}, \ldots, \vc{w}_k \in W$ ЛНеЗ и $\vc{v}_{1}, \ldots, \vc{v}_k \in V,$  & 
                то $\sum_{i=1}^k b(\vc{v}_i, \vc{w}_i) = 0$ $\Longrightarrow$ $\vc{v}_{1}=\ldots =\vc{v}_k=0$;\\
                (Т3) &
                $b$ -- сюръективно, т.е. &
                $T = \langle b(\vc{v}, \vc{w}) \mid \vc{v} \in V, \; \vc{w} \in W\rangle_{\mathbb{F}}$.
             \end{tabular}
      
      Кроме того, пара $(b, T)$ универсальна в том смысле, что какова ни была пара $(b', T')$, состоящая из векторного пространства $T'$ и билинейного отображения $b' \colon V \times W \to T'$, найдётся единственное линейное отображение $\sigma \colon T \to T'$, для которого $b' (\vc{v}, \vc{w}) = \sigma (b(\vc{v}, \vc{w}))$, $\vc{v} \in V$, $\vc{w} \in W$.
\end{to_thr}


% \begin{proof}[$\triangle$]
%     \textit{Точно не для экзамена.}
% \end{proof}

\begin{to_def} 
    Пару $(b, T)$, однозначно определенную с точностью до изоморфизма по заданным векторным пространствам $V$, $W$, называют \textit{тензорным произведением} этих пространств.
\end{to_def}

\begin{to_def} 
    Пусть $\mathcal A \colon V \to V, \mathcal B \colon W \to W$ -- линейный операторы. Их \textit{тензорным произведением} называется линейный оператор 
    $$
        \mathcal A \otimes \mathcal B \colon V \otimes W \to V \otimes W,
    $$
    действующий по правилу
    $$
        \left(\mathcal A \otimes \mathcal B\right) \left(\vc{v} \otimes \vc{w}\right) = \mathcal A \vc{v} \otimes \mathcal B \vc{w}
    $$
    (далее по линейности $\left(\mathcal A \otimes \mathcal B\right) \left( \sum (\vc{v}_i \otimes \vc{w}_i)\right) = \sum \mathcal A \vc{v}_i \otimes \mathcal B \vc{w}_i$).
\end{to_def}

%%%%%%%%%%%%%%%%%%%%%%%%%%%%%%%%%%%%%%%%%%%%%%%%%%%%%%%%%%%%%%%%%%%%%%%%
%%%%%%%%%%%%%%%%%%%%%%%%%%%%%%%%%%%%%%%%%%%%%%%%%%%%%%%%%%%%%%%%%%%%%%%%
\subsection{Свёртка, симметризация и альтернирование тензоров}
\subsubsection{Свёртка}

\begin{to_def}[свёртка]
    Зафиксировав все переменные кроме $\vc{x}_r$ и $u_s$, получим билинейную форму:
    \begin{equation}
        f(\vc{x}_r, u_s) := T(\dots, \vc{x}_r, \dots, u_s, \dots).
    \end{equation}
    Тогда \textbf{инвариантная} сумма вида
    $
        \overline{T} = f(\vc{e}_k, e^k)
    $
    называется \textbf{свёрткой тензора} $T$ по $r$-му ковариантному и $s$-му контрвариантному индексу.
\end{to_def} 

Если обозначить свёртку по индексам $r$, $s$ символом $\tr^s_r$, то $\tr^s_r$ -- линейное отображение:
\begin{equation}
    \tr^s_r: \ten^q_p(V) \to \ten_{p-1}^{q-1}(V).
\end{equation}

\begin{to_thr}
    Свёртка вида $\tr^s_r$ тензора $T \in \ten^q_p$ является тензор $\overline{T} \in \ten^{q-1}_{p-1}$ с координатами
    \begin{equation}
        \overline{T}^{
        j_1, \dots, j_{s-1}, j_{s+1}, \dots, j_q
        }_{
        i_1, \dots, i_{r-1}, i_{r+1}, \dots, i_p
        } = 
        \sum_k T^{
        j_1, \dots, j_{s-1}, k, j_{s+1}, \dots, j_q
        }_{
        i_1, \dots, i_{r-1}, k, i_{r+1}, \dots, i_p
        }
    \end{equation}
\end{to_thr}



%%%%%%%%%%%%%%%%%%%%%%%%%%%%%%%%%%%%%%%%%%%%%%%%%%%%%%%%%%%%%%%%%%%%%%%%
\subsubsection{Симметричные тензоры}
\marginpar{$S\ten^p(V)$}
Для любой перестановки $\pi \in S_p$ положим 
\begin{equation}
    f_{\pi}(T)(\vc{x}_1, \dots, \vc{x}_p) = T(\vc{x}_{\pi(1)}, \dots, \vc{x}_{\pi(p)})
\end{equation}
\begin{to_def}
    Тензор $T$ типа $(p, 0)$ называется \textbf{симметричным}, если $\forall \pi \in S_p$ $f_{\pi}(T) = T$. \textbf{Симметризацией} $T \in \ten^0_p (V)$ называется отображение
    \begin{equation}
        S(T) = \frac{1}{p!} \sum_{\pi \in S_p} f_{\pi} (T) \colon  \ten^0_p (V) \to \ten^0_p (V).
    \end{equation}
\end{to_def}

ЗФ: Подпространство сим. тензоров типа $\ten^0_p (V)$ обозначим $\ten^+_p (V)$.  Действие $S$: 1) $S^2 = S$, $\Im S = \ten^+_p (V)$. Пространства $\F[X_1, \dots, X_n]_p$ и $\ten^+_p (V)$ биективны. Тогда
\begin{equation}
    \dim \F[X_1, \dots, X_n]_p = \dim \ten^+_p (V) = 
    \begin{pmatrix}
        n+p-1 \\ p
    \end{pmatrix}
\end{equation}

\begin{to_def}
    Ассоциативная и коммутативная \textbf{симметрическая алгебра} пространства $V$:
    \begin{equation}
        S(V) = \oplus_{p=0}^{\infty} S\ten^p (V),
    \end{equation}
    где $\vee$ выступает в качестве умножения.

\end{to_def}




%%%%%%%%%%%%%%%%%%%%%%%%%%%%%%%%%%%%%%%%%%%%%%%%%%%%%%%%%%%%%%%%%%%%%%%%
\subsubsection{Кососимметричные тензоры}

\begin{to_def}
    Назовём тензор $T$ кососимметричным, если 
    \begin{equation}
        f_{\pi}(T) = \sign(\pi) \cdot T \hspace{1cm} \forall \pi \in S_p.
    \end{equation}
\end{to_def}

\begin{to_def}
    \textbf{Альтерированием} называется отображение
    \begin{equation}
        A(T) = \frac{1}{p!} \sum_{\pi \in S_p} \sign(\pi) \cdot f_{\pi} (T) \colon  \ten^0_p (V) \to \ten^0_p (V).
    \end{equation}
\end{to_def}


Действие $A$: 1) $A^2 = A$, 2) $\Im A = \Lambda^+_p (V)$, 3) $A(f_{\sigma}(T)) = \sign(\sigma) A(T)$.




%%%%%%%%%%%%%%%%%%%%%%%%%%%%%%%%%%%%%%%%%%%%%%%%%%%%%%%%%%%%%%%%%%%%%%%%
%%%%%%%%%%%%%%%%%%%%%%%%%%%%%%%%%%%%%%%%%%%%%%%%%%%%%%%%%%%%%%%%%%%%%%%%
\subsection{Внешняя алгебра}

\begin{to_def}
    Зададим \textbf{операцию внешнего умножения}
    \begin{equation}
        \wedge \colon \Lambda(V) \times \Lambda(V) \to \Lambda (V),
    \end{equation}
    полагая 
    $
        Q \wedge R = A(Q \otimes R)
    $
    для любого $q$-вектора $Q$ и любого $r$-вектора $R$. 
\end{to_def}

\begin{to_def}[алгебра Грассмана]
    Алгебра $\Lambda (V)$ над $\F$ называется \textbf{внешней алгеброй} пространства $V$:
    \begin{equation}
        \Lambda(V) = \oplus_p^n \Lambda^p (V)
    \end{equation}
\end{to_def}

\begin{to_thr}
    Внешняя алгебра ассоциативна.
\end{to_thr}

\noindent
$\boxed{\Rightarrow}$. Пусть $\vc{x}_1, \vc{x}_2, \dots, \vc{x}_p$ -- произвольные векторы из $V$. Тогда\footnote{
    с. 287, Кострикин.
} 
\begin{equation}
    \vc{x}_1 \wedge \vc{x}_2 \wedge \dots \wedge \vc{x}_p = A(\vc{x}_1 \otimes \vc{x}_2 \otimes \dots \otimes \vc{x}_p).
\end{equation}

\begin{to_thr}
    Пусть $(\vc{e}_1, \dots, \vc{e}_p)$ -- базис $V$. Тогда 
    \begin{equation}
        \vc{e}_1 \wedge \vc{e}_2 \wedge \dots \wedge \vc{e}_p, \hspace{1cm} 1 \leqslant i_1 < \dots < i_p \leqslant n
    \end{equation}
    образуют базис пространства $\Lambda^p(V)$.
\end{to_thr}

\noindent
$\boxed{\Rightarrow}$. Внешняя алгебра $\Lambda(V)$ пространства $V$ имеет размерность $2^n$. При этом 
\begin{equation}
    \dim \Lambda^p (V) = \begin{pmatrix}
        n \\ p
    \end{pmatrix}.
\end{equation}
Базис пространства $\Lambda^n (V)$ состоит из одного $n-$вектора 
$$
    \vc{e}_1 \wedge \vc{e}_2 \wedge \dots \wedge \vc{e}_p.
$$

Внешняя алгебра $V$ \textbf{антикоммутативна}:
\begin{equation}
    Q \in \Lambda^q(V), R \in \Lambda^r(V) \Rightarrow Q \wedge R = (-1)^{qr} R \wedge Q.
\end{equation}


\noindent
\# связь с определителями\\
\# векторные подпространства и $p$-векторы\\
\# условия разложимости $p$-векторов


