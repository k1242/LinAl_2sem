\sbs{\small Интегрируемость по Риману разных функций}

\begin{minipage}[t]{0.45\textwidth}
%%%%%%%%%%%%%%%%%%%%%%%%%%%%%%%%%%%%%%%%%%%%%%%%%%%%%%%%%%%%%%%%%%%%%%%%
\begin{proof}[
thr $\triangle$
\eqref{4.109}]

\phantom{42}

\noindent
1) $\tau \vdash [a, b]$, $\Omega(f, \tau) < \varepsilon$. \\
2) $\omega(f, \Delta) = \sup\{|f(x')-f(x'')| \mid x', x'' \in \Delta\}$, \\
3) $||A|-|B|| \leq |A-B|$ $\leadsto$ $\omega(|f|, \Delta) \leq \omega(f, \Delta)$. \\
4) $\Omega(|f|, \tau) \leq \Omega(f, \tau) < \varepsilon$ $\Rightarrow$ $|f| \in \mathcal{R}$.\\
5) Пусть $f, g \leq M$. If $\Omega(f, \tau) < \varepsilon, \Omega(g, \tau) < \varepsilon,$ то $|AB-CD|\leq|B| \cdot |A-C| + |C| \cdot |B-D|$ $\Rightarrow$ $\Omega(fg, \tau) \leq M \Omega(f, \tau) + M \Omega(g, \tau) \leq 2M \varepsilon$.
\end{proof}

\begin{proof}[
thr $\triangle$
\eqref{4.113} ф-ла Н-Л]

\phantom{42}

\noindent
1) Пусть $\tau:$ $a=x_0 < x_1 < \ldots < x_N = b$; \\
2) по thr Лагранжа: \\
$F(b) - F(a) = \sum_{k=1}^N F(x_{k}) - F(x_{k-1})=$\\
$= \sum_{k=1}^N f(\xi_k) (x_k - x_{k-1}) = \sum_{k=1}^N f(\xi) |\Delta_k|$.
\end{proof}
%%%%%%%%%%%%%%%%%%%%%%%%%%%%%%%%%%%%%%%%%%%%%%%%%%%%%%%%%%%%%%%%%%%%%%%%
\end{minipage}
\hfill
\begin{minipage}[t]{0.45\textwidth}
%%%%%%%%%%%%%%%%%%%%%%%%%%%%%%%%%%%%%%%%%%%%%%%%%%%%%%%%%%%%%%%%%%%%%%%%
\begin{proof}[
\red{thr} $\triangle$
\eqref{4.112}]

\phantom{42}

\noindent
1) Для ступ. Пусть $|h| \leq M$ и $\exists N$ ступ. \\
2) По адд $\int$ $\forall \tau$: \\
$\int_a^b h(x) \d x = \sum_{\Delta \in \tau} \int_{\Delta} h(x) \d x$, \\
$\int_a^b h \d x - \sigma(h, \tau, \xi) = \sum_{\Delta \in \tau} \int_{\Delta} (h(x) - h(\xi(\Delta))) \d x$. \\
3) При $|\tau| < \delta$ слаг $< 2 M \delta$. $\Rightarrow$ \\
откл не более $2 M N \delta$ $\forall \xi$. 
$MN \equiv MN(h) \neq (\Delta)$, то для ступ Q.E.D. \\
4) $f \sim g \leq f$ и $\int_a^b g \d x \geq \int_a^b f \d x - \varepsilon/2$. \\
5) Пусть $g \leq M_g$ и имеет $N_g$ разрывов. \\
при $|\tau| < \frac{\varepsilon}{M_g N_g}$ в силу мон $\sigma$:\\
$\sigma(f, \tau, \xi) \geq \sigma(g, \tau, \xi) > \int_a^b g \d x - \varepsilon/2 \geq \int_a^b f \d x - \varepsilon$. 
Аналогично для $h \geq f$.
\end{proof}
%%%%%%%%%%%%%%%%%%%%%%%%%%%%%%%%%%%%%%%%%%%%%%%%%%%%%%%%%%%%%%%%%%%%%%%%
\end{minipage}

\phantom{42}

\begin{minipage}[t]{0.45\textwidth}
%%%%%%%%%%%%%%%%%%%%%%%%%%%%%%%%%%%%%%%%%%%%%%%%%%%%%%%%%%%%%%%%%%%%%%%%
\begin{proof}[
thr $\triangle$
\eqref{4.113} int -- первообразная]

\phantom{42}

\noindent
1) $\Omega(f, \tau) = \sum_{\Delta \in \tau} \omega(f, \Delta) |\Delta| \leq \omega_f (|\tau|) \sum_{\Delta \in \tau} |\Delta| = \omega_f(|\tau|)(b-a)$. \\
2) $\omega_f(|\tau|) \to 0$ при $|\tau| \to 0$ по равн непр. \\
3) адд, лин $\leadsto$ \\
$F(x + u)  - F(x)  = \int_x^{x+u} f(t) \d t = f(x) u + \int_x^{x+u} (f(t) - f(x))\d t.$ \\
4) $|f(t) - f(x)| \leq \omega_f(|u|)$ $\Rightarrow$ \\
$\big| \int_x^{x+u} f(t) - f(x) \d t
\big| \leq 
\big| \int_x^{x+u} \omega_f (|u|) \d t
\big| \leq \omega_f(|u|)|u| \to 0. 
$
\end{proof}
%%%%%%%%%%%%%%%%%%%%%%%%%%%%%%%%%%%%%%%%%%%%%%%%%%%%%%%%%%%%%%%%%%%%%%%%
\end{minipage}
\hfill
\begin{minipage}[t]{0.45\textwidth}
%%%%%%%%%%%%%%%%%%%%%%%%%%%%%%%%%%%%%%%%%%%%%%%%%%%%%%%%%%%%%%%%%%%%%%%%
\begin{proof}[
thr $\triangle$
\eqref{4.113}мон $\to$ $\in \mathcal{R}$]

\phantom{42}

\noindent
1) $f \leq M$, \\
2) $\sum_{\Delta \in \tau} \omega(f, \Delta) \leq \omega(f, [a, b]) = |f(b) - f(a)|.$ \\
3) $\Omega(f, \tau) = \sum_{\Delta \in \tau} \omega(f, \Delta) |\Delta| \leq |\tau| \sum_{\Delta \in \tau} \omega(f, \Delta) \leq |\tau| \cdot |f(b) - f(a)| \to 0$.
\end{proof}
%%%%%%%%%%%%%%%%%%%%%%%%%%%%%%%%%%%%%%%%%%%%%%%%%%%%%%%%%%%%%%%%%%%%%%%%
\end{minipage}