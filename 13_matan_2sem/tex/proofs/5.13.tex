\sbs{\small Примеры применения интеграла Лебега}

\begin{minipage}[t]{0.45\textwidth}
%%%%%%%%%%%%%%%%%%%%%%%%%%%%%%%%%%%%%%%%%%%%%%%%%%%%%%%%%%%%%%%%%%%%%%%%
\begin{proof}[
{thr} $\triangle$
\eqref{5.89}]

\phantom{42}

\noindent
1) При $\int_X |f(x)|^2 \d x = 0$ $\triangle \dots \square$.\\
2) $f, g$ $\leadsto$ $\int|f|^2 \d x = \int|g|^2 \d x = 1$. \\
3) $|fg| \leq |f|^2/2 + |g|^2/2$. \\
4) $\int (3) \leadsto$ $\int_X |fg| \d x \leq 1 \Rightarrow 
\big| \int_X fg \d x
\big| \leq 1.$


\end{proof}


\begin{proof}[
{thr} $\triangle$
\eqref{5.90}]

\phantom{42}

\noindent
1) $\lr{\int}'$ -- lim + линейность: \\
$\lim\limits_{h \to 0} \frac{1}{h}\lr{
    \int_X f(x, y + h) \d x - \int_X f(x, y) \d x
} = $ \\
$= \lim\limits_{h \to 0} \int_X \frac{1}{h} \lr{f(x, y+h) - f(x, y)} \d x$. \\
\small *) Поточ к $lim$ под $\int$+thr об огр.сход, т.к. (2).\\
\normalsize
2) $\big| 
\frac{1}{h} \lr{f(x, y+h) - f(x, y)} 
\big| =$ \\
\phantom{42}
\hfill $ = |f'_y(x, y + \vartheta(x)h)| \leq g(x).$
\end{proof}


\begin{proof}[
{thr} $\triangle$
\eqref{5.91}]

\phantom{42}
\noindent

1) Let $g>0, \, m = \inf \{ f \mid x \in X \}, \, M = \sup \dots$;\\
2)инт-м $m \leq f(x) \leq M:$\\ $m \int_X g(x) \d x \leq \int_X f(x) g(x) \d x \leq M \int_X g(x) \d x$;\\
3) $f$-непр на связном $X$, $\Im f \supset (m,M)$;\\
4) < у (2): $\xi:$ $f(\xi) \int_X g(x) \d x = \int_X f(x) g(x) \d x$
5) =(2): $\int_X (f(x)-m) g \d x = 0 \leadsto \xi:$ $f(\xi) = m;$

\end{proof}
%%%%%%%%%%%%%%%%%%%%%%%%%%%%%%%%%%%%%%%%%%%%%%%%%%%%%%%%%%%%%%%%%%%%%%%%
\end{minipage}
\hfill
\begin{minipage}[t]{0.5\textwidth}
%%%%%%%%%%%%%%%%%%%%%%%%%%%%%%%%%%%%%%%%%%%%%%%%%%%%%%%%%%%%%%%%%%%%%%%%
\begin{proof}[
\red{thr} $\triangle$
\eqref{5.92}]

\phantom{42}
\noindent
1) $g(a+0):= g(a)$, $g(b-0) = g(b)$;\\
2) $f\to f_n$ в ср.непр.диф. $f_n$: $\int_a^b|f-f_n|\d x < \frac{1}{n}$;\\
3) для $f_n$ доказанно из огр. $g\leq M$, $\iiint < \frac{3M}{n}:$\\ 
$\left| \int_a^b f q \d x - g(a) \int_a^{\vartheta_n} f \d x - g(b) \int_{\vartheta_n}^b f \d x \right| < 3M/n$;\\
4) по th.Больц-Вей $\vartheta_n \to \vartheta$, $n\to \infty$ для равенства в пределе;\\
Теперь $f$ ограничена, проделаем то же самое с $g$:\\
5) как и с $f$: $\int_a^b |g(x) - g_n(x)| \d x < \frac{1}{n}$, $g_n(a) \to g(a)$;\\
6) В итоге требуется док. для непр.диф. $f_n,g_n$;\\
7) Начнём: $g(b) = 0\leadsto \int_a^b f g \d x = g(a) \int_a^{\vartheta} f \d x?$;\\
8)$F = \int_a^x f \d t$,$\int_a^b f g \d x = -\int_a^b F g' \d x$($F(a), g(b) = 0$);\\
9)Let $g' \leq 0$ \eqref{5.91} $\exists \vartheta:$ $\int_a^b F(-g') \d x = F(\vartheta) g(a)$;\\
10) В общ.лучае $g \to g(x) - g(b):$
\begin{equation*}
	\begin{aligned}
		\int_a^b f g \d x = \int_a^b f (g(x) - g(b)) \d x + \int_a^b f g(b) \d x =\\
		= (g(a)-g(b)) \int_a^{\vartheta} f \d x + g(b) \int_a^b f \d x = \\
		= g(a) \int_a^{\vartheta} f \d x + g(b)  \int_{\vartheta}^b f \d x.  
	\end{aligned}
\end{equation*}
\end{proof}
%%%%%%%%%%%%%%%%%%%%%%%%%%%%%%%%%%%%%%%%%%%%%%%%%%%%%%%%%%%%%%%%%%%%%%%%
\end{minipage}