\sbs{\small Интеграл Лебега для ступенчатых и произвольных функций}

\begin{minipage}[]{0.45\textwidth}
%%%%%%%%%%%%%%%%%%%%%%%%%%%%%%%%%%%%%%%%%%%%%%%%%%%%%%%%%%%%%%%%%%%%%%%%
\begin{proof}[
lem $\triangle$
\eqref{5.58}]

\phantom{42}

\noindent
1) $\sigma(X')= X''$, $f(X_i') = c_i'$ и $f(X_j'') = c_j''$;\\
2) $\sum\nolimits_{ i }^{  } c_i'\mu(X_i') = \sum\nolimits_{ i,j }^{  } c_i'\mu(X_i'\cap X_j'') =$\\$\sum\nolimits_{ i,j }^{} c_j''\mu(X_i'\cap X_j'') = \sum\nolimits_{ j }^{  } c_j'' X_j''$;\\
3)$\sum \sum < \infty: \sum_+ + \sum_-$;\\
4)thr. Фубини для каждой.
\end{proof}
%%%%%%%%%%%%%%%%%%%%%%%%%%%%%%%%%%%%%%%%%%%%%%%%%%%%%%%%%%%%%%%%%%%%%%%%
\end{minipage}
\hfill
\begin{minipage}[]{0.55\textwidth}
%%%%%%%%%%%%%%%%%%%%%%%%%%%%%%%%%%%%%%%%%%%%%%%%%%%%%%%%%%%%%%%%%%%%%%%%
\begin{proof}[
 lem$ \triangle$
\eqref{5.59}]

\phantom{42}

\noindent
1) $\mu(X)< \infty$: $X_i = \{ x \in X \mid \varepsilon i \leq f(x) < \varepsilon (i+1) \}$;\\
2) $g(X_i) = \varepsilon i$, $h(X_i) = \varepsilon (i+1)$;\\
3) Очев. $g \leq f \leq h$ и $\int_X (h(x) - g(x))\d x = \varepsilon \mu(X)$;\\
4) $\mu(X) \to \infty$: $X\overset{2^{-i}\varepsilon}{\to} Y_i$(cчёт), $\mu(Y_i)< \infty$. 

\end{proof}
%%%%%%%%%%%%%%%%%%%%%%%%%%%%%%%%%%%%%%%%%%%%%%%%%%%%%%%%%%%%%%%%%%%%%%%%
\end{minipage}

\begin{minipage}[]{0.49\textwidth}
%%%%%%%%%%%%%%%%%%%%%%%%%%%%%%%%%%%%%%%%%%%%%%%%%%%%%%%%%%%%%%%%%%%%%%%%
\begin{proof}[
lem $\triangle$
\eqref{5.62}]

\phantom{42}

\noindent
1) $\forall \varepsilon > 0$, по \eqref{5.59}, $\exists g \leq f \leq h$: 
$\int_X(h-g)\d x < \varepsilon$. \\ 
2) $g^* = \max\{g_0, g\}$, $h^* = \min\{h_0, h\}$ \\ 
3) $h^* - g^* \leq h - g$, $g_0 \leq g^* \leq f \leq h^* \leq h_0$\\ 
4) $\int_X h^* \d x < \int_X g^* \d x + \varepsilon$.
\end{proof}
%%%%%%%%%%%%%%%%%%%%%%%%%%%%%%%%%%%%%%%%%%%%%%%%%%%%%%%%%%%%%%%%%%%%%%%%
\end{minipage}
\hfill
\begin{minipage}[]{0.49\textwidth}
%%%%%%%%%%%%%%%%%%%%%%%%%%%%%%%%%%%%%%%%%%%%%%%%%%%%%%%%%%%%%%%%%%%%%%%%
\begin{proof}[
 lem $\triangle$
\eqref{5.65}]

\phantom{42}

\noindent
1) $\forall \varepsilon > 0$, по \eqref{5.59}, $\exists g \leq f \leq h$: 
$\int_X(h-g)\d x < \varepsilon$. \\ 
2) if $\int g = + \infty$ — противоречие, else: \\ 
$h = g + (h-g)$ $\Rightarrow$ $(\forall \varepsilon)$ $\overline{\int} = \underline{\int}$.

\end{proof}
%%%%%%%%%%%%%%%%%%%%%%%%%%%%%%%%%%%%%%%%%%%%%%%%%%%%%%%%%%%%%%%%%%%%%%%%
\end{minipage}

\begin{minipage}[]{0.45\textwidth}
%%%%%%%%%%%%%%%%%%%%%%%%%%%%%%%%%%%%%%%%%%%%%%%%%%%%%%%%%%%%%%%%%%%%%%%%
\begin{proof}[
thr $\triangle$
\eqref{5.66}]

\phantom{42}

\noindent
1) $\Rightarrow:$ прибл. ступ $g \leq f \leq h$, $\therefore \forall x \in X:$ $f_+ (x) \leq h_+(x), \text{ и } f_-(x) \geq g_-(x)$;\\
2) из (\ref{4.25}) $\int_X f_+(x) \d x \leq \int_X h_+(x) \d x < + \infty$ и \\
$\int_X f_-(x) \d x \geq \int_X g_- (x) \d x > -\infty$;\\
3) $\Leftarrow:$ сост-ть $f$ из прибл. ступ. $f_-$ и $f_+$.

\end{proof}
%%%%%%%%%%%%%%%%%%%%%%%%%%%%%%%%%%%%%%%%%%%%%%%%%%%%%%%%%%%%%%%%%%%%%%%%
\end{minipage}
\hfill
\begin{minipage}[]{0.47\textwidth}
%%%%%%%%%%%%%%%%%%%%%%%%%%%%%%%%%%%%%%%%%%%%%%%%%%%%%%%%%%%%%%%%%%%%%%%%
\begin{proof}[
 thr $\triangle$
\eqref{5.69}]

\phantom{42}

\noindent
1) $\boxed{\Rightarrow}$. $Z \in \bigcup_{A_n}$, где $A_n = \{x\in I \mid \omega(f, x) \geq \frac{1}{n}\}$ \\
2) Дост $\mu(A_n)=0$. fix $\delta > 0$.
$\forall A_n \; \exists \{Q_i\} \colon A_n \in \bigcup_i^N Q_i$ 
$\sum_i \mu(Q^*_i) < \delta$;
$\Rightarrow$ $\mu(A_n) = 0$ $\Rightarrow$ $\mu(Z) = 0$. \\
3) $\boxed{\Leftarrow}$. $\mu(Z) = 0 \Rightarrow \exists \{Q_i\}
\colon \sum_i \mu(Q_i) < \varepsilon_1$ \\
4) $X \setminus \bigcup_i Q_i \subseteq X \setminus Z$ $\Rightarrow$ $\forall x \in X \setminus \bigcup_i Q_i$ $\omega(f, x) < \varepsilon_2$;\\
5) $\exists Q(x) \colon \omega(f, Q) < \varepsilon_3$;
6) $\exists \mathcal{B}$ -- кон. покр. $X$. 
7) $\sum_j \omega(f, D_j) \cdot \xi_{D_J} (x) = \varphi(x)$.
\end{proof}
%%%%%%%%%%%%%%%%%%%%%%%%%%%%%%%%%%%%%%%%%%%%%%%%%%%%%%%%%%%%%%%%%%%%%%%%
\end{minipage}