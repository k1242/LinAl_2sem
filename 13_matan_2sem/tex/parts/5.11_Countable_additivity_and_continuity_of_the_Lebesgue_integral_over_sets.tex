\sbs{Счётная аддитивность и непрерывность интеграла Лебега по множествам}

\begin{to_thr}[Счётная аддитивность интеграла Лебега по множествам]
\label{5.78}
    Если измеримая функция $f \colon X \to \mathbb{R}$ неотрицательна или имеет конечный интеграл на множестве $X$, которое представляется в виде объединения попарно не пересекающихся измеримых по Лебегу множеств как $X = \bigsqcup_{k=1}^{\infty} Y_k$, то
    $$
        \int_X f(x) \d x = \sum_{k=1}^{\infty} \int_{Y_k} f(x) \d x.
    $$
\end{to_thr}

\begin{to_thr}[Непрерывность интеграла Лебега по множествам]
\label{cont_L}
    Пусть множества $X_k \subseteq \mathbb{R}^n$ измеримы по Лебегу и возрастают по включению
    $$
    X_1 \subseteq X_2 \subseteq X_3 \subseteq \dots \subseteq X_k \subseteq X_{k_1} \subseteq \dots.
    $$
    Пусть также $X = \bigcup_K X_k$ и интеграл Лебега $\int_X f(x) \d x$ конечен или $f$ измерима и неотрицательна. Тогда
    $$
    \int_X f(x) \d x = \lim_{k \to \infty} \int_{X_k} f(x)\d x.
    $$
\end{to_thr}

\begin{to_thr}[Непрерывность интеграла Лебега по убыванию множеств]
\label{5.81}
    Пусть множества $X_k \subseteq \mathbb{R}^n$ измеримы по Лебегу и убыввают по включению
    $$
    X_1 \supseteq X_2 \supseteq X_3 \supseteq \dots \supseteq X_k \supseteq X_{k_1} \supseteq \dots.
    $$
    Пусть также $X = \bigcap_K X_k$ и интеграл Лебега $\int_X f(x) \d x$ конечен \sout{или $f$ измерима и неотрицательна}. Тогда
    $$
    \int_X f(x) \d x = \lim_{k \to \infty} \int_{X_k} f(x)\d x.
    $$
\end{to_thr}

\begin{to_thr}[Непрерывность\footnote{
    Имеет место абсолютная непрерывность интеграла Лебега.
} интеграла Лебега по верхнему пределу]
\label{5.82}
    Если $f \in \L_c$ на $[a, b]$, то непрерывно зависит от $x \in [a, b]$
    $$
    g(x) = \int_a^x f(t) \d t.
    $$
\end{to_thr}

