\sbs{Примеры применения интеграла Лебега}

\begin{to_thr}[Неравенство Коши-Буняковского]
\label{5.89}
    Пусть функции $f, g \colon X \to \mathbb{R}$ измеримы по Лебегу и их $|f|^2$, $|g|^2$ имеют конечные интегралы. Тогда
    $$
    \lr{\int_X f(x) g(x) \d x}^2 \leq 
    \lr{\int_X |f(x)|^2 \d x} \cdot \lr{\int_X |g(x)|^2 \d x}.
    $$
\end{to_thr}

\begin{to_thr}[Дифференцирование\footnote{
    Функция $g \colon X \to \mathbb{R}^+$ и $g \in \L_c$
} под знаком интеграла]
\label{5.90}
    \begin{equation*}
    \begin{split}
    \left.
        \begin{aligned}
            &f(x, y) \in \L^x_c \; \forall y \in (a, b) \\
            &f \text{ дифференцируема по } y \\
            &\forall x \in X, \forall y \in (a, b) |f'_y(x, y)| \leq g(x) \\
            &g \geq 0 \colon X \to \mathbb{R}^+ \in L_c \text{ на } X
        \end{aligned}
    \right\} \hspace{1cm}
    \Longrightarrow \hspace{1cm}
    \frac{d}{dy} \int_X f(x, y) \d x = \int_X f'_y (x, y) \d x.
    \end{split}
    \end{equation*}
    \end{to_thr}

\begin{to_thr}[Теорема о среднем для интеграла]
\label{5.91}
    \begin{equation*}
    \begin{split}
    \left.
        \begin{aligned}
            &f \colon X \to \mathbb{R} \text{ непрерывна} \\
            &f \text{ ограничена на связном } X\\
            &g \geq 0 \colon X \to \mathbb{R}^+ \in L_c \text{ на } X
        \end{aligned}
    \right\} \hspace{1cm}
    \Longrightarrow \hspace{1cm}
    \exists \xi \in X \colon \;\; 
    \int_X f(x) g(x) \d x = f(\xi) \int_X g(x) \d x.
    \end{split}
    \end{equation*}
\end{to_thr}

\begin{to_thr}[Вторая о среднем для интеграла]
\label{5.92}
    \begin{equation*}
    \begin{split}
    \left.
        \begin{aligned}
            &f \in \L_c \\
            &g \text{ монотонна} \\
            &g \text{ ограничена}
        \end{aligned}
    \right\} \hspace{0.5cm}
    \Longrightarrow \hspace{0.5cm} \exists \vartheta \in [a, b] \colon
    \int_a^b f(x) g(x) \d x = 
    g(a + 0) \int_a^{\vartheta} f(x) \d x + 
    g(b - 0) \int_{\vartheta}^b f(x) \d x.
    \end{split}
    \end{equation*}
\end{to_thr}



