\sbs{Линейность и монотонность интеграла Лебега}

\begin{to_thr}
\label{5.70}
    Если интегралы $\int_X f_1 (x) \d x$ и $\int_X f_2 (x) \d x$ определены и конечны, а $A, B \in \mathbb{R}$, то 
    $$
        \int_X (Af_1 + B f_2) \d x = A \int_X f_1 \d x + B \int_X f_2 \d x.
    $$
\end{to_thr}

\begin{to_thr}
\label{5.71}
    Если функция $f \geq 0 \in \L$ на $X \colon \mu(X) > 0$, то\footnote{
        $\int_X f(x) \d x = 0$ тогда, и только тогда, когда $f(x) = 0$ на всём $X$, кроме множества лебеговой меры нуль. 
    } 
    $$
        \int_X f(x) \d x \geq 0.
    $$
\end{to_thr}

\begin{to_thr}
\label{5.72}
    $f \in \L_c$ обязана быть измеримой по Лебегу.
\end{to_thr}