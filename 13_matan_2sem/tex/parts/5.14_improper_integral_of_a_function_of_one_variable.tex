\sbs{Несобственный интеграл функции одной переменной}

\begin{to_def}
    \textbf{Несобственный интеграл} на $(a, b)$ с особенностью в $b$ называется предел 
    $$
        \int_a^{*b} f(x) \d x = \lim_{\beta \to b-0} \int_a^{\beta} f(x) \d x 
    $$
    в предположении, что все интегралы $\int_a^{\beta} f(x) \d x$ конечны.
\end{to_def}

В случае существования интеграла Лебега по свойству \textbf{непрерывной зависимости} интеграла от верхнего предела несобственный интеграл оказывается равным собственному. При этом в силу свойства \textbf{абсолютной сходимости} интеграла Лебега интеграл от $|f|$ также будет конечен. Поэтому содержательный случай несобственного
интеграла — это \textit{условная сходимость}.

\begin{to_thr}[Критерий Коши сходимости несобственного интеграла]
\label{5.96}
    Несобственный интеграл $\int_a^{*b} f(x) \d x$ сходится тогда, и только тогда, когда
    $$
        \forall \varepsilon > 0 \; \exists \beta (\varepsilon) \; \forall \xi, \eta \in [\beta(\varepsilon), b), 
        \bigg| \int_{\xi}^{\eta} f(x) \d x
        \bigg| < \varepsilon.
    $$
\end{to_thr}

\begin{to_thr}[Признак Дирихле]
\label{5.97}
    Пусть на $[a, +\infty]$ $f(x)$ непрерывна, имеет ограниченную первообразную \textbf{и} $g(x)$ является монотонной и $\lim_{x \to \infty} g(x) = 0$. Тогда
    $$
        \int_a^{\infty} f(x) g(x) \d x \text{ --- сходится.}
    $$
\end{to_thr}

\begin{to_thr}[Признак Абеля]
\label{5.98}
    \textbf{Если} функция $f(x)$ непрерывна на $[a, b)$ и интеграл $f(x)$ сходится, \textbf{и} функция $g(x)$ ограниченна и монотонна на $[a, b)$, \textbf{то} сходится интеграл $\int_a^b f(x) g(x) dx$.
\end{to_thr}

\begin{to_thr}[Интегральный признак сходимости числового ряда]
\label{5.100}
    Пусть $f \colon [1, +\infty) \to \mathbb{R}^+$ убывает. Тогда
    $$
    \sum_{n=1}^{\infty} f(n) < + \infty \hspace{1cm} \Leftrightarrow \hspace{1cm} \int_1^{+\infty} f(x) \d x < + \infty.
    $$
\end{to_thr}