\sbs{Элементарные множества и мера Жордана}

Назовём \textit{параллелепипедом} $P \supseteq \mathbb{R}^n$ произведение ограниченных промежутков $\Delta_1 \times \dots \times \Delta_n$ (которые могут быть точками). 

% КОРРЕКТНОСТЬ ОПРЕДЕЛЕНИЯ МЕРЫ ЭЛЕМЕНТАРНЫХ МНОЖЕСТВ
\begin{to_thr}[\hyperlink{cor_link}{Корректность определения меры элементарных множеств}]
\label{cor42}
    Мера элементарного множества не зависит от его представления в виде объединения параллелепипедов. 
\end{to_thr}

% АДДИТИВНОСТЬ МЕРЫ ЭЛЕМЕНТАРНЫХ МНОЖЕСТВ
\begin{to_thr}[\href{https://youtu.be/OlE7PruV6nY?list=PLocvKxfon41Xqv5Gamwh1aSSc9e97NET0&t=370}
{Аддитивность меры элементарных множеств}]
\label{add}
    Для всяких двух элементарных множеств $S$ и $T$ множества $S \cap T$, $S \cup T$, $S \setminus T$ тоже элементарны и выполняется равенство:
    $$
        mS + mT = m (S \cup T) + m (S \cap T).
    $$
\end{to_thr}

% НИЖНЯЯ МЕРА ЖОРДАНА
\begin{to_def}[\href{https://youtu.be/OlE7PruV6nY?list=PLocvKxfon41Xqv5Gamwh1aSSc9e97NET0&t=1032}
{Нижняя мера Жордана}]
    Нижняя мера Жордана $X$ -- точная верхняя грань меры элементарного множества $s \subseteq X$; верхняя мера Жордана $X$ -- точная нижняя грань меры элементарного множества: $X \subseteq S$. Множество \textbf{измеримо по Жордану} если $\overline{m} = \underline{m}$.
\end{to_def}

\begin{to_thr}[Критерий измеримости множества по Жордану]
    Множество $X \subseteq \mathbb{R}^n$ измеримо по Жордану тогда, и только тогда, когда оно ограничено и $m(\partial X) = 0$.
\end{to_thr}

