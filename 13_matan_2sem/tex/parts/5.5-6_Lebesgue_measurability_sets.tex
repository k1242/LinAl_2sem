\sbs{Измеримость по Лебегу множеств с некоторыми свойствами}

\begin{to_thr}
\label{meas_1}
     Объединение и пересечение счётного числа измеримых по Лебегу множеств измеримо. 
\end{to_thr}

\begin{to_thr}[\hyperlink{meas_2_link}{Непрервность меры Лебега}]
\label{meas_2}
    Если множество $X$ является объединением возрастающей последовательнсоти измеримых множеств
    $$X_1 \subseteq X_2 \subseteq \dots \subseteq X_k \subseteq \dots,$$
    то $lim_{k \to \infty} \mu(X_k) = \mu(X)$, где предел понимается в топологии расширенной числовой прямой. 
\end{to_thr}

\begin{to_thr}
\label{meas_3}
    Открытые и замкнутые множества в $\mathbb{R}^n$ измеримы по Лебегу. 
\end{to_thr}

\begin{to_thr}[\hyperlink{meas_4_link}{Внешняя и внутренняя регулярность меры Лебега}]
\label{meas_4}
    Измеримое по Лебегу множество $X$ можно сколь угодно точно по мере приблизить содержащим его открытым множеством. Если $X$ имеет конечную меру, то его можно сколь угодно точно по мере приблизить содержащимся в нём компактным множеством.
\end{to_thr}



