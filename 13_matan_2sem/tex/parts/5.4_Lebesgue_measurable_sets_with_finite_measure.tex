\sbs{Измеримые по Лебегу множества с бесконечной мерой}

\begin{to_def}
    Множество $X \subseteq \mathbb{R}^n$ называется \textbf{измеримым по Лебегу с бесконечной мерой}, если его можно представить в виде счётного объединения попарно не пересекающихся множеств
    $$
        X = \bigsqcup_{k=1}^{\infty} X_k,
    $$
    каждое из которых измеримо по Лебегу с конченой мерой и $\sum_{k=1}^{\infty} \mu (X_k) = + \infty$.

    \label{5.22d}
\end{to_def}

% Множество называется измеримым по Лебегу, если оно измеримо по
% Лебегу с конечной мерой или с бесконечной мерой.

\begin{to_thr}
    Измеримость множества по Лебегу сохраняется при взятии конечных объединений, пересечений и разности множеств.
    \label{5.34t}
\end{to_thr}

\begin{to_thr}[Счётная аддитивность меры Лебега в общем случае]
    Если множества $X_k, k \in \mathbb{N}$, попарно не пересекаются и измеримы по Лебегу, то их объединение измеримо по Лебегу и выполняется
    $$
    \mu\lr{\bigsqcup_{k=1}^{\infty} X_k} = \sum_{k=1}^{\infty} \mu (X_k).
    $$
    где считаем, что наличие в сумме хотя бы одного бесконечного слагаемого делает сумму
    бесконечной.
    \label{5.35t}
\end{to_thr}