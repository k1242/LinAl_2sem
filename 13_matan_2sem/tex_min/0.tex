% \documentclass[a4papper,12pt]{article}

% здесь очень явно не хватает формул типа интеграла пуассона


\documentclass[twoside]{article}

\usepackage[T2A]{fontenc}
\usepackage[utf8]{inputenc}
\usepackage[english,russian]{babel} 
\usepackage{amsmath} 
\usepackage{
amssymb,textcomp, esvect,esint}
\usepackage[margin=1in]{geometry}
\usepackage{titling} 
\usepackage{amsfonts}
\usepackage{amsthm}
\usepackage{graphicx}
\usepackage{indentfirst} 
\usepackage{xcolor} 
\usepackage{enumitem}
\usepackage[unicode, pdftex]{hyperref}
\usepackage{booktabs}
\usepackage{caption}
\usepackage{listings}
\usepackage{multirow}
\usepackage{pifont}
\usepackage{fancyhdr}                       %   добавить верхний и нижний колонтитул

\renewcommand{\Im}{\mathop{\mathrm{Im}}\nolimits}
\renewcommand{\Re}{\mathop{\mathrm{Re}}\nolimits}
\renewcommand{\d}{\, d}
\renewcommand{\leq}{\leqslant}
\renewcommand{\geq}{\geqslant}
\newcommand{\vc}[1]{\mbox{\boldmath $#1$}}
\newcommand{\com}[1]{\textcolor{red}{#1}}
\newcommand{\T}{^{\text{T}}}
\newcommand{\const}{\text{const}}
\newcommand{\lr}[1]{\left(#1\right)}
\newcommand{\oo}{\mathop{\mathrm{o}}\nolimits}
\newcommand{\OO}{\mathop{\mathrm{O}}\nolimits}
\newcommand{\xmark}{\ding{55}}


\newtheorem{to_def}{DEF}
\newtheorem{to_thr}{THR}
\newtheorem{to_suj}{SUJ}
\newtheorem{to_lem}{LEM}
\newtheorem{to_com}{COM}

\definecolor{linkcolor}{HTML}{0000CC}
\definecolor{urlcolor}{HTML}{0000CC}
\hypersetup{pdfstartview=FitH,  linkcolor=linkcolor,urlcolor=urlcolor, colorlinks=true}


\pagestyle{fancy}
\fancyhf{}
\fancyhead[RE,LO]{\textsc{Ф\raisebox{-1.5pt}{и}з\TeX}}
\fancyhead[LE,RO]{МатАн (min)}
\fancyfoot[LE,RO]{\textcolor{grey}{\texttt{\thepage}}}


\begin{document}
\setlength{\abovedisplayskip}{5pt}
\setlength{\abovedisplayshortskip}{5pt}
\setlength{\belowdisplayskip}{5pt}
\setlength{\belowdisplayshortskip}{5pt}

\thispagestyle{plain}
\begin{center}
    \LARGE \textsc{ТеорМин по \textbf{математическому анализу} II}
\end{center}

\hrule

\phantom{42}

\begin{flushright}
    \begin{tabular}{rr}
    % written by:
        \textbf{Авторы}: 
        & Хоружий Кирилл \\
        & Примак Евгений \\
        &\\
    % date:
        \textbf{От}: &
        \textit{26.05.2020}\\
    \end{tabular}
\end{flushright}



\section*{Ряды с неотрицательными членами}

\noindent
\textbf{\textit{Критерий сходимости ряда с неотрицательными членами}}:
$$
\sum_{n=1}^{\infty} a_n \; \;(a_n \geqslant 0, n \in \mathbb{N})\text{ сходится }  \Leftrightarrow \exists M > 0 \colon \forall n \in \mathbb{N} \; \sum_{k=1}^n a_k \leqslant M.
$$

\noindent
\textbf{\textit{Интегральный признак сходимости ряда}}:\\
\begin{tabular}{rllll}
Если $f(x)\, :$ & неотрицательна           & то&  $\sum_{n=1}^{\infty} f(n)$ 
& \multirow{2}{*}{сходятся и расходятся одновременно.}\\
                & убывает на $[1, +\infty)$&   &  $\int_{n=1}^{\infty} f(x) \, dx$
& \\
\end{tabular}

\phantom{42}

\noindent
\textbf{\textit{Признак\footnote{
    Пример для $\lambda=1$: $a_n = n^{-1}$ и $a_n = n^{-2}$, расходится и сходится соответсвенно.
} Даламбера}}:
\begin{equation*}
        \text{Если } a_n > 0 \; (n \in \mathbb{N}) \text{\textbf{ и }} 
\exists \lim_{n \to \infty} \frac{a_{n+1}}{a_n} = \lambda, \text{ то } 
        \left\{
        \begin{aligned}
            &{
            \text{сходится}
            }  ,&{  \text{при } 
            \lambda < 1;
            }\\
            &{
            \text{расходится}
            }  ,&{ \text{при }
            \lambda > 1.
            }
        \end{aligned}
        \right.
\end{equation*}

\noindent
\textbf{\textit{Признак Коши}}:
\begin{equation*}
        \text{Если } a_n > 0 \; (n \in \mathbb{N}) \text{\textbf{ и }} 
\exists \lim_{n \to \infty} \sqrt[n]{a_n} = \lambda, \text{ то } 
        \left\{
        \begin{aligned}
            &{
            \text{сходится}
            }  ,&{  \text{при } 
            \lambda < 1;
            }\\
            &{
            \text{расходится}
            }  ,&{ \text{при }
            \lambda > 1.
            }
        \end{aligned}
        \right.
\end{equation*}

\noindent
\textbf{\textit{Признак Раабе}}:
\begin{equation*}
        \text{Если } a_n > 0 \; (n \in \mathbb{N}) \text{\textbf{ и }} 
\exists \lim_{n \to \infty} 
n \lr{\frac{a_n}{a_{n+1}} - 1}
= \lambda, \text{ то } 
        \left\{
        \begin{aligned}
            &{
            \text{сходится}
            }  ,&{  \text{при } 
            \lambda < 1;
            }\\
            &{
            \text{расходится}
            }  ,&{ \text{при }
            \lambda > 1.
            }
        \end{aligned}
        \right.
\end{equation*}

\noindent
\textbf{\textit{Признак\footnote{
    Поддразумевается $|\gamma_n| < c$, $\delta > 0$. Второй случай верен при $\alpha = 1$.
} Гаусса}}:
\begin{equation*}
        \text{Если } \dots \text{\textbf{ и }} 
\frac{a_n}{a_{n+1}} = \alpha + \frac{\beta}{n} + \frac{\gamma_n}{n^{1 + \delta}}, \text{ то } 
        \left\{
        \begin{aligned}
            &{
            \text{сход}
            }  ,&{  \text{при } 
            \alpha < 1;
            }\\
            &{
            \text{расход}
            }  ,&{ \text{при }
            \alpha > 1.
            }
        \end{aligned}
        \right.;
        \left\{
        \begin{aligned}
            &{
            \text{сход}
            }  ,&{  \text{при } 
            \beta > 1;
            }\\
            &{
            \text{расход}
            }  ,&{ \text{при }
            \beta \leqslant 1.
            }
        \end{aligned}\right.
\end{equation*}


\section*{Знакопеременный ряд}

\noindent
\textbf{\textit{Признак Лейбница}}:

Если ряд $(-1)^n|\dots|$ \textbf{и} члены ряда монотонно убывают $\lim\limits_{n\to \infty} |a_n| = 0$, то ряд сходится.





\section*{Абсолютно и не абсолютно сходящиеся ряды}

\noindent
\textbf{\textit{Признак Дирихле}}:
$$
\sum_{n=1}^{\infty} a_n b_n \text{ сходится, если } 
\exists M > 0 \; \; \forall n \in \mathbb{N} \colon 
\bigg| \sum_{k=1}^n b_k \bigg| \leqslant M \text{ \textbf{и} }
\lim_{n \to \infty} a_n = 0.
$$

\noindent
\textbf{\textit{Признак Абеля}}:
$$
\sum_{n=1}^{\infty} a_n b_n \text{ сходится, если } 
(a_n) \text{ монотонна и ограниченна \textbf{и} } 
\sum_{n=1}^{\infty} b_n \text{ сходится.}
$$

\begin{to_thr}[теорема Римана]
    Если ряд сходится условно, то $\forall A$ существует такая перестановка членов ряда, что сумма полученного ряда равна $A$.
\end{to_thr}

\section*{Сходимость и равномерная сходимость функциональных рядов}

\begin{to_def}
    Последовательность функций $f_n \colon X \to \mathbb{R}$ \textbf{сходится равномерно} на множестве $X$ к функции $f_0$, если
    $$
\sup \{
 |f_n(x) - f_0(x)| \mid x \in X   
\} \to 0, \; \; \; n \to \infty.
    $$ Также обозначается как $f_n \rightrightarrows_X f_0$.
\end{to_def}    

\begin{to_def}
    Ряд из функций $u_n \colon X \to \mathbb{R}$ \textbf{сходится равномерно}\footnote{
        Или, $\forall \varepsilon > 0 \; \exists N  \; \forall n > N \; \forall x \in E \colon |S_n(x) - f(x)| < \varepsilon$.
    } на $X$, если последовательность его частичных сумм
    $$
        S_n(x) = \sum_{k=1}^{\infty} u_k (x)
    $$
    сходится равномерно на $X$ к некоторой функции $S \colon X \to \mathbb{R}$.
\end{to_def}


\noindent
\textbf{\textit{Критерий Коши}} (равномерной сходимости \textbf{последовательности функций}):

Для последовательности функций $f_n \colon X \to \mathbb{R}$ выполняется
$$
\forall \varepsilon > 0 \; \exists N (\varepsilon)  \; \forall n, m \geq N(\varepsilon), \; \sup\{
|f_n(x) - f_m(x)| \colon x \in X
    \} < \varepsilon
$$
тогда, и только тогда, когда последовательность равномерно на $X$ сходится к некоторой функции.

%%%%%%%%%%%%%%%%%%%%%%%%%%%%%%%%%%%%%%%%%%%%%%%%%%%%%%%%%%%%%%%%%%%%%%%%

\phantom{42}

\noindent
\textbf{\textit{Критерий Коши}} (равномерной сходимости \textbf{функционального ряда}):

Для функционального ряда $\sum_{n=1}^{\infty} f_n \colon X \to \mathbb{R}$ выполняется
$$
\forall \varepsilon > 0 \; \exists N (\varepsilon)  \; \forall n, m \colon m \geq n > N(\varepsilon), \; \forall x \in E :
 \bigg|\sum_{k=n}^m f_k (x) \bigg| < \varepsilon.
$$
тогда, и только тогда, когда функциональный ряд   сходится на $X$ равномерно к некоторой функции.

%%%%%%%%%%%%%%%%%%%%%%%%%%%%%%%%%%%%%%%%%%%%%%%%%%%%%%%%%%%%%%%%%%%%%%%%

% \noindent
% \# необходимое условие сходимости функционального ряда ($\mathbb{K}$: с. 102).

\begin{to_thr}[Признак Вейерштрасса]
\textbf{Если} ряд из функций $u_n : X \to \mathbb{R}$, $\sum_{n=1}^\infty u_n(x)$, $\forall x \in X \forall n \in \mathbb{N}, |u_n(x)| \leq a_n$ и $\sum_{n=1}^\infty a_n < + \infty $, \textbf{то} ряд из функций сходится равномерно и абсолютно на $X$.

\end{to_thr}

%%%%%%%%%%%%%%%%%%%%%%%%%%%%%%%%%%%%%%%%%%%%%%%%%%%%%%%%%%%%%%%%%%%%%%%%

\begin{to_thr}[Непрерывность равномерного предела непрерывных функций] 
    \textbf{Пусть} последовательность $f_n \colon X \to \mathbb{R}$ сходится равномерно на $X$ к функции $f$ и все функции $f_n$ непрерывны. \textbf{Тогда} $f$ тоже непрерывна.
\end{to_thr}

\begin{to_thr}
    \textbf{Пусть} последовательность дифференцируемых функций $f_n \colon [a, b] \to \mathbb{R}$ сходится в точке $x_0$, а последовательность производных $f'_n$ сходится равномерно на $[a, b]$ к функции $g$. \textbf{Тогда} $(f_n)$ равномерно сходится к некоторой $f$ и $f'=g$ на всём отрезке $[a, b]$.
\end{to_thr}     
\section*{Равномерная сходимость несобственных интегралов}

\begin{to_def}
	Интеграл $\int_a^{+ \infty}f(x;\alpha) \d x$, сходящийся для $\forall \alpha \in E$, называют \textbf{равномерно сходящимся на множестве $E$}, если для $\forall \varepsilon > 0 \, \exists \delta_\varepsilon: \forall \alpha \in E$ \textbf{и} $\xi \geq \delta_\varepsilon$ выполняется:
	$$\left|\int_a^{+\infty} f(x; \alpha)\d x \right| < \varepsilon.$$
\end{to_def}

\noindent \textbf{\textit{Признак Вейерштрасса}}

Если на $[\alpha; +\infty)$ $\exists \varphi(x): |f(x;\alpha)|\leq \varphi$ (для $\forall x \in [\alpha, +\infty)$ и $\forall \alpha \in E$), \textbf{и}
если интеграл $\int_a^{+\infty} \varphi (x) \d x$ сходится, то интеграл $\int_a^{+ \infty}f(x;\alpha) \d x$ сходится \textbf{абсолютно} и \textbf{равномерно} на $E$.

\phantom{239}


\noindent 
\textbf{\textit{Признак Дирихле}}

 Интеграл $\int_\alpha^{+\infty} f(x; \alpha) g(x; \alpha) \d x$ сходится равномерно по $\alpha$ на $E$, если при каждом фиксированном $\alpha \in E$ функции $f, g, g_x'$ непрерывны по $x$ на $[\alpha; +\infty)$ \textbf{и} удовлетворяют следующим условиям:
 \begin{enumerate}
 	\item $g(x;\alpha) \to 0$ при $x \to + \infty$ равномерно относительно $\alpha \in E$;
 	\item $g_x'(x; \alpha)$ для каждого фиксированного $\alpha \in E$ не меняет знака  $\forall x \in  [\alpha; +\infty)$;
 	\item $\forall \alpha \in E$ $f$ имеет ограниченную первообразную.
 \end{enumerate}


\noindent \textbf{\textit{Критерий Коши}}

$\int_a^{+ \infty}f(x;\alpha) \d x$ сходится равномерно на $E$ $\Longleftrightarrow$ $\forall \varepsilon > 0 \; \exists \delta_\varepsilon \in (\alpha; + \infty): \forall \xi' \in [\delta_\varepsilon; + \infty), \xi'' \in [\delta_\varepsilon; +\infty)$ и $\forall \alpha \in E$ выполняется: $\left| \int_{\xi'}^{\xi''}f(x;\alpha)\d x \right| < \varepsilon$.


\phantom{239}

\noindent \textbf{\textit{Непрерывность равномерно сходящегося интеграла по параметру}}

Если $f(x;\alpha)$ непрерывна на множестве и $I(\alpha) = \int_\alpha^{+ \infty}$ сходится равномерно по $\alpha$ на $[\alpha_1; \alpha_2]$, \textbf{то} $I(\alpha)$ непрерывна на $[\alpha_1; \alpha_2]$.
\section*{Свойства интегрируемости}

\begin{to_thr}
	Пусть $g \leq 0$, $g$ интегрируема по Лебегу на $X$. Если $|f| \geq g$ почти всюду на $X$ $\Rightarrow$ $f$ интегрируема на $X$:
	$$\int_X f(x) \, \d \mu \leq \int_X g(x) \d \mu$$
\end{to_thr}


\begin{to_thr}
	Пусть $\mu (X) < + \infty$, $f:X \rightarrow \mathbb{R}$ -- измерима и ограничена $\Rightarrow$ $f$ интегрируема на $X$.
\end{to_thr}

\begin{to_thr}
	Пусть $f:X \rightarrow \mathbb{R}$ -- непрерывна на компакте $X$, $\Rightarrow$ $f$ интегрируема на $X$.
\end{to_thr}

\begin{to_thr}[Неравенство Чебышёва]
	$f \colon X \rightarrow \mathbb{R}$ интегрируема по Лебегу. Тогда $\forall C \in \mathbb{R}$ $\exists$ измеримый $X_C = \{ x \in X \colon |f(x)|\geq C \}$, $f$ интегрируема на $X_C$. Выполняется неравенство:
	$$ \int_X |f(x)| \d \mu \geq \int_{X_C} |f(x)| \d \mu \geq C \mu(X_C) $$
\end{to_thr}
\section*{Сходимость интеграла знакопостоянной\footnote{
    Кроме средней колонки, там $\checkmark_{\text{\xmark}}$ -- условная сходимость.
} функции}

\begin{minipage}[]{0.25\textwidth}
\begin{equation*}
\begin{split}
%%%%%%%%%%%%%%%%%%%%%%%%%%%%%%%%%%%%%%%%%%%%%%%%%%%%%%%%%%%%%%%%%%%%%%%%
        \int_0^1 \frac{\d x}{x^{\alpha}}
        &\left\{\begin{aligned}
            &{
                \checkmark
            }  ,&{
                \alpha < 1
            }\\
            &{
                \text{\xmark}
            }  ,&{
                \alpha \geq 1
            }
        \end{aligned} \right. \\
        %%%%%%%%%%%%%%%%%%%%%%%%%%%%%%%%%%%%%%%%%%%%%%%%%%%%%%%%%%%%%%%%
        \int_1^{\infty} \frac{\d x}{x^{\alpha}}
        &\left\{\begin{aligned}
            &{
                \checkmark
            }  ,&{
                \alpha > 1
            }\\
            &{
                \text{\xmark}
            }  ,&{
                \alpha \leq 1
            }
        \end{aligned} \right.  \\
        %%%%%%%%%%%%%%%%%%%%%%%%%%%%%%%%%%%%%%%%%%%%%%%%%%%%%%%%%%%%%%%%
        \int_0^{\infty} \frac{\d x}{e^{\alpha x}}
        &\left\{\begin{aligned}
            &{
                \checkmark
            }  ,&{
                \alpha > 0
            }\\
            &{
                \text{\xmark}
            }  ,&{
                \alpha \leq 0
            }
        \end{aligned} \right. 
%%%%%%%%%%%%%%%%%%%%%%%%%%%%%%%%%%%%%%%%%%%%%%%%%%%%%%%%%%%%%%%%%%%%%%%%
\end{split}
\end{equation*}
\end{minipage}
\hfill
\begin{minipage}[]{0.35\textwidth}
\begin{equation*}
\begin{split}
%%%%%%%%%%%%%%%%%%%%%%%%%%%%%%%%%%%%%%%%%%%%%%%%%%%%%%%%%%%%%%%%%%%%%%%%
        \int_1^{\infty} \frac{\sin^2 x}{x^{\alpha}} \d x
        &\left\{\begin{aligned}
            &{
                \checkmark
            }  ,&{
                \alpha > 1.
            }\\
            &{
                \text{\xmark}
            }  ,&{
                \alpha \leq 1
            }
        \end{aligned} \right. \\
        %%%%%%%%%%%%%%%%%%%%%%%%%%%%%%%%%%%%%%%%%%%%%%%%%%%%%%%%%%%%%%%%
        \int_1^{\infty} \frac{\sin x}{x^{\alpha}} \d x
        &\left\{\begin{aligned}
            &{
                \checkmark_{\text{\xmark}}
            }  ,&{
                \alpha > 1.
            }\\
            &{
                \checkmark
            }  ,&{
                \alpha \in (0, 1]
            }\\
            &{
                \text{\xmark}
            }  ,&{
                \alpha \leq 0
            }
        \end{aligned} \right. \\
        %%%%%%%%%%%%%%%%%%%%%%%%%%%%%%%%%%%%%%%%%%%%%%%%%%%%%%%%%%%%%%%%
%%%%%%%%%%%%%%%%%%%%%%%%%%%%%%%%%%%%%%%%%%%%%%%%%%%%%%%%%%%%%%%%%%%%%%%%
\end{split}
\end{equation*}
\end{minipage}
\hfill
\begin{minipage}[]{0.35\textwidth}
\begin{equation*}
\begin{split}
%%%%%%%%%%%%%%%%%%%%%%%%%%%%%%%%%%%%%%%%%%%%%%%%%%%%%%%%%%%%%%%%%%%%%%%%
        \int_0^{0.5} \frac{\d x}{x^{\alpha} |\ln x|^{\beta}}
        &\left\{\begin{aligned}
            &{
                \checkmark
            }  ,&{
                \alpha < 1, \, \forall \beta;\; a=1, \beta > 1
            }\\
            &{
                \text{\xmark}
            }  ,&{
                \alpha > 1, \, \forall \beta;\; a=1, \beta \leq 1
            }
        \end{aligned} \right. \\
        %%%%%%%%%%%%%%%%%%%%%%%%%%%%%%%%%%%%%%%%%%%%%%%%%%%%%%%%%%%%%%%%
        \int_2^{\infty} \frac{\d x}{x^{{\alpha}} \ln^{\beta} x}
        &\left\{\begin{aligned}
            &{
                \checkmark
            }  ,&{
                \alpha > 1, \, \forall \beta;\; a=1, \beta > 1
            }\\
            &{
                \text{\xmark}
            }  ,&{
                \alpha < 1, \, \forall \beta;\; a=1, \beta \leq 1
            }
        \end{aligned} \right.  \\
        %%%%%%%%%%%%%%%%%%%%%%%%%%%%%%%%%%%%%%%%%%%%%%%%%%%%%%%%%%%%%%%%
        \int_1^{+\infty} \frac{\d x}{e^{\alpha} x^{\beta}}
        &\left\{\begin{aligned}
            &{
                \checkmark
            }  ,&{
                \alpha > 0, \, \forall \beta;\; a=0, \beta > 1
            }\\
            &{
                \text{\xmark}
            }  ,&{
                \alpha < 0, \, \forall \beta;\; a=0, \beta \leq 1
            }
        \end{aligned} \right. 
%%%%%%%%%%%%%%%%%%%%%%%%%%%%%%%%%%%%%%%%%%%%%%%%%%%%%%%%%%%%%%%%%%%%%%%%
\end{split}
\end{equation*}
\end{minipage}


\section*{Признаки сходимости несобственных интегралов}

\begin{to_thr}[критерий Коши сходимости несобственного интеграла первого рода]
    Для того, чтобы несобственный интеграл сходился, необходимо и достаточно, чтобы было выполнено (условие Коши): $\forall \varepsilon > 0 \; \exists A > a \colon \forall A'>A$ и $\forall A'' > A$ верно:
    $$
    \bigg|
    \int_{A'}^{A''} f(x) \d x
    \bigg| < \varepsilon.
    $$
\end{to_thr}

\begin{to_thr}[н\&д условие сходимости несобственного  интеграла от неотрицательной функции]
    Если $f(x) \geq 0$ на $[a, +\infty)$, то для сходимости н\&д, чтобы функция
    $$
    \Phi (A) = \int_a^A f(x) \d x
    $$
    была огарниченной на $A \in [a, +\infty)$.
\end{to_thr}

% \noindent
% \# \textit{признак сравнения} \\
% \# \textit{признак сравнения в предельной форме}.

% \phantom{42}

\noindent
\textbf{\textit{Признак Дирихле}}:

Пусть на $[a, +\infty]$ $f(x)$ непрерывна, имеет ограниченную первообразную \textbf{и} $g(x)$ является монотонной, непрерывно дифференцируемой на $[a, \infty)$ и $\lim_{x \to \infty} g(x) = 0$. Тогда
$$
\int_a^{\infty} f(x) g(x) \d x \text{ --- сходится.}
$$

\noindent
\textbf{\textit{Признак Абеля}}:

\textbf{Если} функция $f(x)$ непрерывна на $[a, b)$ и интеграл $f(x)$ сходится, \textbf{и} функция $g(x)$ ограниченна, непрерывно дифференицруема и монотонна на $[a, b)$, \textbf{то} сходится интеграл $\int_a^b f(x) g(x) dx$.
\section*{Вычисление объёмов тел и площадей поверхности}

Для вращения относительно $Ox$ непрерывной функции, с непрерывной неотрицательной производной $f(t)$ (для II случая), верно, что:

\begin{table}[h]
    \centering
        \begin{tabular}{r|c|c}
    % \toprule
            & \centering $V$ & $S$ \\
    \midrule
            $y(x)$ \phantom{$\dfrac{1}{2}$}
            & $\pi \int_a^b y^2 (x) \, dx$
            & $2 \pi \int_a^b |y(x)| \sqrt{1 + y'^2 (x)} dx$\\
            $x(t), y(t)$ \phantom{$\dfrac{1}{2}$}
            & $\pi \int_a^b y^2 (x) \, dx$
            & $ 2 \pi \int_{\alpha}^{\beta} |y(t)| \sqrt{x'^2(t) + y'^2(t)} dt$  \\
    % \bottomrule
        \end{tabular}
\end{table}

Площадь, при вращении вокруг полярного луча кривой $r = r(\varphi)$, $0 \leqslant \varphi_1 \leqslant \varphi_2 \leqslant \pi/2$:
$$
S = 2 \pi \int_{\varphi_1}^{\varphi_2} r(\varphi) \sqrt{r^2(\varphi) + r'^2(\varphi)} \cos{\varphi} \, d \varphi.
$$
\section*{Дифференцируемые функции нескольких переменных}
\begin{to_def}
	Функция $f:U(x_0, y_0) \rightarrow \mathbb{R}$ называется \textit{дифференцируемой} в точке $(x_0, y_0)$, если она представима в виде:
	$$f(x,y) = f(x_0, y_0) + A (x - x_0) + B (y - y_0) + \oo(\rho)$$
	$$d f_(x_0, y_0) := A \Delta x + B \Delta y = A \, dx + B \, d y$$
\end{to_def}

\begin{to_thr}
	Если $f$ дифференцируема в точке $(x_0, y_0)$ и $\d f (x_0, y_0) = A \d x + B \d y$, \textbf{то} в этой точке существуют частные производные функции: $f_x' = A;\, f_y' = B$.
\end{to_thr}


\begin{to_thr}[Достаточное условие]
	Частные производные $f$ существуют и непрерывны в точке $(x_0, y_0)$ $\Rightarrow$ $f$ дифференцируема в этой точке.
	\label{suff_cond_diff}
\end{to_thr}

\noindent Алгоритм исследования функции на дифференцируемость в особой точке $(x_0, y_0)$ (не попадающей под теорему \ref{suff_cond_diff}):
\begin{enumerate}
	\item Ищем в $(x_0, y_0)$ частные производные. Если $ \nexists \Rightarrow f$ не дифференцируема;
	\item Ищем\footnote{
	Что то же самое, что и 
	$f(\rho \cos \varphi, \rho \sin \varphi) \rightrightarrows A$.
	} 
	$ \lim_{\rho \to 0}\lr{ \frac{1}{\rho} \bigg|f(x, y) - f(x_0, y_0) - f_x'\big|_{(x_0,y_0)} (x - x_0) - f_y'\big|_{(x_0,y_0)} (y - y_0)\bigg|}$,
	если предел существует и равен нулю, \textbf{то} $f$ дифференцируема, \textbf{иначе}  нет.
\end{enumerate}

\newpage

\section*{Ряд Тейлора}

\begin{table}[h]
	\centering
	\caption{Формулы Маклорена для элементарных функций}

	\begin{tabular}{c|c}
	\phantom{$\dfrac{42}{42}$} $e^x$ & $1 + x + \frac{x^2}{2 !} + \ldots + \frac{x^n}{n !} + \oo (x^n)$ \\

	\phantom{$\dfrac{42}{42}$}$\sin x$ & $x - \frac{x^3}{3!} + \frac{x^5}{5 !} + \ldots + \frac{(-1)^n x^{2n+1}}{(2n + 1) !} + \oo(x^{2n+2})$ \\
 	
 	\phantom{$\dfrac{42}{42}$}$\cos x$ & $1 - \frac{x^2}{2 !} + \frac{x^4}{4 !} + \ldots + \frac{(-1)^n x^{2n}}{(2n)!} + \oo(x^{2n+1})$ \\

 	\phantom{$\dfrac{42}{42}$}$\tg x$& $x + \frac{x^3}{3} + \frac{2x^5}{15} + \frac{17 x^7}{315} + \ldots$ \\

	\phantom{$\dfrac{42}{42}$} $\ctg x$& $\frac{1}{x} - \frac{x}{3} - \frac{x^3}{45} - \frac{2x^5}{945} - \ldots$ \\ 	
	
	\phantom{$\dfrac{42}{42}$}$\sh x$ & $x + \frac{x^3}{3 !} + \frac{x^5}{5 !} + \ldots + \frac{x^{2n+1}}{(2n + 1) !} + \oo(x^{2n+2})$ \\
 	
 	\phantom{$\dfrac{42}{42}$}$\ch x$ & $1 + \frac{x^2}{2 !} + \frac{x^4}{4 !} + \ldots + \frac{x^{2n}}{(2n)!} + \oo(x^{2n+1})$\\

 	\phantom{$\dfrac{42}{42}$} $\arcsin x$ & $x + \frac{x^3}{2 \cdot 3} + \frac{1 \cdot 3 x^5}{2 \cdot 4 \cdot 5} + \ldots + \frac{1 \cdot 3 \cdot 5 \ldots (2n -1) x^{2n+1}}{2 \cdot 4 \ldots (2n)(2n+1)} + \oo(x^{2n+2})$ \\

 	\phantom{$\dfrac{42}{42}$}$\arctg x$ & $x - \frac{x^3}{3} + \frac{x^5}{5} - \frac{x^7}{7} + \ldots + \frac{(-1)^n x^{2n+1}}{2n+1} + \oo (x^{2n+2})$ \\
 	
 	\phantom{$\dfrac{42}{42}$}$(1 + x)^\alpha$ & $1 + \alpha x + \frac{\alpha (\alpha - 1) }{2 !}x^2 + \ldots + \frac{\alpha (\alpha - 1) \ldots (\alpha - (n-1))}{n!}x^n + \oo(x^n)$ \\

 	\phantom{$\dfrac{42}{42}$}$ln(1 + x)$ & $x - \frac{x^2}{2} + \frac{x^3}{3 !} + \ldots + \frac{(-1)^{n-1} x^n}{n} + \oo(x^n)$ \\
	\end{tabular}
\end{table}

\noindent
\textbf{\textit{Формула Коши-Адамара}}:
$$
\frac{1}{R} = \varlimsup_{n \to \infty} |a_n|^{1/n}, \hspace{1cm} 
\lr{R = \lim_{n \to \infty} \frac{a_n}{a_{n+1}} \; \; \text{если существует}}
$$


\noindent
\textbf{\textit{Формула Тейлора для $f(x, y)$}}:
$$
f(x, y) = \sum_{k=0}^m \frac{1}{k!} \sum_{i=0}^k \frac{k!}{(k-i)!i!} \frac{\partial^k f(x_0, y_0)}{\partial x^{k-i} \partial y^i} (x - x_0)^{k-i} (y - y_0)^i + \oo(\rho^m),
$$
где 
$$
\rho = \sqrt{(x-x_0)^2 + (y-y_0)^2}, \; \; x \to x_0, \; \; y \to y_0.
$$
В частности:
$$
f(x, y) = f + f'_x x + f'_y y + \frac{1}{2} \lr{f''_{xx} x^2  + 2 f''_{xy} xy + f''_{yy} }
$$


\subsection*{Бонус}

\noindent
\textbf{\textit{Интеграл неразложимой рациональной функции}}:
$$
\int \frac{Ax + B}{x^2 + 2px + q} \d x = \frac{A}{2} \ln{\lr{x^2 + 2px + q}} + \frac{B - Ap}{\sqrt{q - p^2}} \arctg{\frac{x + p}{q - p^2}}
$$

\noindent
\textbf{\textit{Неравенство с логарифмом}}:
$$
\frac{x}{1 + x} \leq \ln\lr{1+x} \leq x.
$$

\newpage

\section*{Примеры}

\subsection*{Ступенчатый ответ}
\begin{minipage}[]{0.3\textwidth}
Пусть $f$ и $g$ интегрируемы на отрезке $[1, a]$
для любого числа $a > 1$. Обозначим \\
\phantom{42} \hspace{1cm} $I_1 = \int_1^{\infty} f(x) dx$, \\
\phantom{42} \hspace{1cm} $I_2 = \int_1^{\infty} g(x) dx$, \\
\phantom{42} \hspace{1cm} $I_3 = \int_1^{\infty} g(x) f(x) dx$. \\
\textbf{Возможно, что}:
\end{minipage}
\hfill
\begin{minipage}[]{0.3\textwidth}
\begin{equation*}
\begin{split}
%%%%%%%%%%%%%%%%%%%%%%%%%%%%%%%%%%%%%%%%%%%%%%%%%%%%%%%%%%%%%%%%%%%%%%%%
\Delta_n &= \left[n, n + \frac{1}{n^4} \right]; \; \; \Delta = \bigcup_{n=1}^{\infty} ; \\
f(x) &= \left\{
    \begin{aligned}
        &n^2, &x \in \Delta_n; \\
        &0, &x \in R \setminus \Delta;
    \end{aligned}
    \right. \\
g(x) &= \left\{
    \begin{aligned}
        &(-1)^n n, &x \in \Delta_n ;\\
        &0, &x \in R \setminus \Delta;
    \end{aligned}
    \right.
%%%%%%%%%%%%%%%%%%%%%%%%%%%%%%%%%%%%%%%%%%%%%%%%%%%%%%%%%%%%%%%%%%%%%%%%
\end{split}
\end{equation*}
\end{minipage}
\hfill
\begin{minipage}[]{0.2\textwidth}
\begin{equation*}
\begin{split}
%%%%%%%%%%%%%%%%%%%%%%%%%%%%%%%%%%%%%%%%%%%%%%%%%%%%%%%%%%%%%%%%%%%%%%%%
\checkmark I_1 &= \sum_{n=1}^{\infty} \frac{1}{n^2}; \\
\checkmark I_2 &= \sum_{n=1}^{\infty} \frac{(-1)^n}{n^3}; \\
\checkmark_{\text{\xmark}} I_3 &= \sum_{n=1}^{\infty} \frac{(-1)^n}{n}.
%%%%%%%%%%%%%%%%%%%%%%%%%%%%%%%%%%%%%%%%%%%%%%%%%%%%%%%%%%%%%%%%%%%%%%%%
\end{split}
\end{equation*}
\end{minipage}


\phantom{42}


\noindent
Бонус:
\begin{equation*}
    \frac{\sin x}{x} \checkmark_{\text{\xmark}} \hspace{1cm}
    \frac{\sin x}{\sqrt{x}} \checkmark_{\text{\xmark}} \hspace{1cm}
    \frac{\cos x}{\sqrt x} \checkmark_{\text{\xmark}} \hspace{1cm}
    x^3 \sin(x^5) \checkmark_{\text{\xmark}}
\end{equation*}


\end{document}