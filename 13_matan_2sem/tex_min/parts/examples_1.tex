\section*{Примеры}

\subsection*{Ступенчатый ответ}
\begin{minipage}[]{0.3\textwidth}
Пусть $f$ и $g$ интегрируемы на отрезке $[1, a]$
для любого числа $a > 1$. Обозначим \\
\phantom{42} \hspace{1cm} $I_1 = \int_1^{\infty} f(x) dx$, \\
\phantom{42} \hspace{1cm} $I_2 = \int_1^{\infty} g(x) dx$, \\
\phantom{42} \hspace{1cm} $I_3 = \int_1^{\infty} g(x) f(x) dx$. \\
\textbf{Возможно, что}:
\end{minipage}
\hfill
\begin{minipage}[]{0.3\textwidth}
\begin{equation*}
\begin{split}
%%%%%%%%%%%%%%%%%%%%%%%%%%%%%%%%%%%%%%%%%%%%%%%%%%%%%%%%%%%%%%%%%%%%%%%%
\Delta_n &= \left[n, n + \frac{1}{n^4} \right]; \; \; \Delta = \bigcup_{n=1}^{\infty} ; \\
f(x) &= \left\{
    \begin{aligned}
        &n^2, &x \in \Delta_n; \\
        &0, &x \in R \setminus \Delta;
    \end{aligned}
    \right. \\
g(x) &= \left\{
    \begin{aligned}
        &(-1)^n n, &x \in \Delta_n ;\\
        &0, &x \in R \setminus \Delta;
    \end{aligned}
    \right.
%%%%%%%%%%%%%%%%%%%%%%%%%%%%%%%%%%%%%%%%%%%%%%%%%%%%%%%%%%%%%%%%%%%%%%%%
\end{split}
\end{equation*}
\end{minipage}
\hfill
\begin{minipage}[]{0.2\textwidth}
\begin{equation*}
\begin{split}
%%%%%%%%%%%%%%%%%%%%%%%%%%%%%%%%%%%%%%%%%%%%%%%%%%%%%%%%%%%%%%%%%%%%%%%%
\checkmark I_1 &= \sum_{n=1}^{\infty} \frac{1}{n^2}; \\
\checkmark I_2 &= \sum_{n=1}^{\infty} \frac{(-1)^n}{n^3}; \\
\checkmark_{\text{\xmark}} I_3 &= \sum_{n=1}^{\infty} \frac{(-1)^n}{n}.
%%%%%%%%%%%%%%%%%%%%%%%%%%%%%%%%%%%%%%%%%%%%%%%%%%%%%%%%%%%%%%%%%%%%%%%%
\end{split}
\end{equation*}
\end{minipage}


\phantom{42}


\noindent
Бонус:
\begin{equation*}
    \frac{\sin x}{x} \checkmark_{\text{\xmark}} \hspace{1cm}
    \frac{\sin x}{\sqrt{x}} \checkmark_{\text{\xmark}} \hspace{1cm}
    \frac{\cos x}{\sqrt x} \checkmark_{\text{\xmark}} \hspace{1cm}
    x^3 \sin(x^5) \checkmark_{\text{\xmark}}
\end{equation*}
