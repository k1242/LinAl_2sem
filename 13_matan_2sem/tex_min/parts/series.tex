\section*{Ряды с неотрицательными членами}

\noindent
\textbf{\textit{Критерий сходимости ряда с неотрицательными членами}}:
$$
\sum_{n=1}^{\infty} a_n \; \;(a_n \geqslant 0, n \in \mathbb{N})\text{ сходится }  \Leftrightarrow \exists M > 0 \colon \forall n \in \mathbb{N} \; \sum_{k=1}^n a_k \leqslant M.
$$

\noindent
\textbf{\textit{Интегральный признак сходимости ряда}}:\\
\begin{tabular}{rllll}
Если $f(x)\, :$ & неотрицательна           & то&  $\sum_{n=1}^{\infty} f(n)$ 
& \multirow{2}{*}{сходятся и расходятся одновременно.}\\
                & убывает на $[1, +\infty)$&   &  $\int_{n=1}^{\infty} f(x) \, dx$
& \\
\end{tabular}

\phantom{42}

\noindent
\textbf{\textit{Признак\footnote{
    Пример для $\lambda=1$: $a_n = n^{-1}$ и $a_n = n^{-2}$, расходится и сходится соответсвенно.
} Даламбера}}:
\begin{equation*}
        \text{Если } a_n > 0 \; (n \in \mathbb{N}) \text{\textbf{ и }} 
\exists \lim_{n \to \infty} \frac{a_{n+1}}{a_n} = \lambda, \text{ то } 
        \left\{
        \begin{aligned}
            &{
            \text{сходится}
            }  ,&{  \text{при } 
            \lambda < 1;
            }\\
            &{
            \text{расходится}
            }  ,&{ \text{при }
            \lambda > 1.
            }
        \end{aligned}
        \right.
\end{equation*}

\noindent
\textbf{\textit{Признак Коши}}:
\begin{equation*}
        \text{Если } a_n > 0 \; (n \in \mathbb{N}) \text{\textbf{ и }} 
\exists \lim_{n \to \infty} \sqrt[n]{a_n} = \lambda, \text{ то } 
        \left\{
        \begin{aligned}
            &{
            \text{сходится}
            }  ,&{  \text{при } 
            \lambda < 1;
            }\\
            &{
            \text{расходится}
            }  ,&{ \text{при }
            \lambda > 1.
            }
        \end{aligned}
        \right.
\end{equation*}

\noindent
\textbf{\textit{Признак Раабе}}:
\begin{equation*}
        \text{Если } a_n > 0 \; (n \in \mathbb{N}) \text{\textbf{ и }} 
\exists \lim_{n \to \infty} 
n \lr{\frac{a_n}{a_{n+1}} - 1}
= \lambda, \text{ то } 
        \left\{
        \begin{aligned}
            &{
            \text{сходится}
            }  ,&{  \text{при } 
            \lambda < 1;
            }\\
            &{
            \text{расходится}
            }  ,&{ \text{при }
            \lambda > 1.
            }
        \end{aligned}
        \right.
\end{equation*}

\noindent
\textbf{\textit{Признак\footnote{
    Поддразумевается $|\gamma_n| < c$, $\delta > 0$. Второй случай верен при $\alpha = 1$.
} Гаусса}}:
\begin{equation*}
        \text{Если } \dots \text{\textbf{ и }} 
\frac{a_n}{a_{n+1}} = \alpha + \frac{\beta}{n} + \frac{\gamma_n}{n^{1 + \delta}}, \text{ то } 
        \left\{
        \begin{aligned}
            &{
            \text{сход}
            }  ,&{  \text{при } 
            \alpha < 1;
            }\\
            &{
            \text{расход}
            }  ,&{ \text{при }
            \alpha > 1.
            }
        \end{aligned}
        \right.;
        \left\{
        \begin{aligned}
            &{
            \text{сход}
            }  ,&{  \text{при } 
            \beta > 1;
            }\\
            &{
            \text{расход}
            }  ,&{ \text{при }
            \beta \leqslant 1.
            }
        \end{aligned}\right.
\end{equation*}


\section*{Знакопеременный ряд}

\noindent
\textbf{\textit{Признак Лейбница}}:

Если ряд $(-1)^n|\dots|$ \textbf{и} члены ряда монотонно убывают $\lim\limits_{n\to \infty} |a_n| = 0$, то ряд сходится.





\section*{Абсолютно и не абсолютно сходящиеся ряды}

\noindent
\textbf{\textit{Признак Дирихле}}:
$$
\sum_{n=1}^{\infty} a_n b_n \text{ сходится, если } 
\exists M > 0 \; \; \forall n \in \mathbb{N} \colon 
\bigg| \sum_{k=1}^n b_k \bigg| \leqslant M \text{ \textbf{и} }
\lim_{n \to \infty} a_n = 0.
$$

\noindent
\textbf{\textit{Признак Абеля}}:
$$
\sum_{n=1}^{\infty} a_n b_n \text{ сходится, если } 
(a_n) \text{ монотонна и ограниченна \textbf{и} } 
\sum_{n=1}^{\infty} b_n \text{ сходится.}
$$

\begin{to_thr}[теорема Римана]
    Если ряд сходится условно, то $\forall A$ существует такая перестановка членов ряда, что сумма полученного ряда равна $A$.
\end{to_thr}
