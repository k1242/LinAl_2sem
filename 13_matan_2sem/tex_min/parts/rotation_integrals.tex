\section*{Вычисление объёмов тел и площадей поверхности}

Для вращения относительно $Ox$ непрерывной функции, с непрерывной неотрицательной производной $f(t)$ (для II случая), верно, что:

\begin{table}[h]
    \centering
        \begin{tabular}{r|c|c}
    % \toprule
            & \centering $V$ & $S$ \\
    \midrule
            $y(x)$ \phantom{$\dfrac{1}{2}$}
            & $\pi \int_a^b y^2 (x) \, dx$
            & $2 \pi \int_a^b |y(x)| \sqrt{1 + y'^2 (x)} dx$\\
            $x(t), y(t)$ \phantom{$\dfrac{1}{2}$}
            & $\pi \int_a^b y^2 (x) \, dx$
            & $ 2 \pi \int_{\alpha}^{\beta} |y(t)| \sqrt{x'^2(t) + y'^2(t)} dt$  \\
    % \bottomrule
        \end{tabular}
\end{table}

Площадь, при вращении вокруг полярного луча кривой $r = r(\varphi)$, $0 \leqslant \varphi_1 \leqslant \varphi_2 \leqslant \pi/2$:
$$
S = 2 \pi \int_{\varphi_1}^{\varphi_2} r(\varphi) \sqrt{r^2(\varphi) + r'^2(\varphi)} \cos{\varphi} \, d \varphi.
$$