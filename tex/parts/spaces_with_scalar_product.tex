\section{Пространства со скалярным произведением}

\subsection{Евклидово пространство}

\begin{minipage}[t]{0.4\textwidth}
    \begin{to_def}[\textbf{Скалярное произведение}]
    Отображение $V \times V \to \mathbb{R}$:\\
        \begin{tabular}{r|ll}
            i) & $(\vc{x}, \vc{y}) = (\vc{y}, \vc{x})$ & $\forall \vc{x}, \vc{y} \in V$; \\
            ii) & $(\alpha \vc{x} + \beta \vc{y}, \vc{z}) = \alpha \Sp{x}{z} + \beta \Sp{y}{z}$ & $\forall \alpha, \beta \in \mathbb{R}$; \\
            iii) & $\Sp{x}{x} > 0$ & $\forall \vc{x} \neq 0$. \\
        \end{tabular}
    \end{to_def}
\end{minipage}
\hfill
\begin{minipage}[t]{0.45\textwidth}
    \noindent
    Любое евклидово пространство содержит:\\
    \begin{tabular}{rl}
        \textit{норма }              & $||\vc{v}|| = \sqrt{\Sp{v}{v}}$ \\
        \textit{угол }               & $\cos \alpha_{{x}}^{{y}} = \Sp{x}{y} / ||\vc{x}|| / ||\vc{y}||$ \\
        \textit{н-во Коши-Буняк.}    & $|\Sp{x}{y}| \leqslant \norm{x} \cdot \norm{y}$\\
        \textit{н-во треугольника}   & $||{\vc{x} \pm \vc{y}}|| \leqslant \norm{x} + \norm{y}$\\
    \end{tabular}
\end{minipage}


\begin{to_def} 
  \textit{Евклидовым векторным пространством} называется вещественное векторное пространство $V$ с выделенной на нём симметричной билинейной формой $(\vc{x}, \vc{y}) \mapsto (\vc{x} | \vc{y})$ такой, что соответствующая квадратичная форма $\vc{x} \mapsto (\vc{x} | \vc{x})$  положительна определена.
\end{to_def}


\subsubsection{Процесс ортогонализации}

\begin{to_def} 
    Базис $(\vc{e}_{1}, \ldots, \vc{e}_n)$ евклидова векторного пространства $V$ называется \textit{ортогональным}, если $(\vc{e}_i | \vc{e}_j) = 0$ при $i \neq f$; $i, j = 1, 2, \ldots, n$. Если, кроме того, $(\vc{e}_i | \vc{e}_i) = 1$, то базис называется \textit{ортонормированным}. 
\end{to_def}

Факт: любые ненулевые взаимно ортогональные векторы $\vc{e}_{1}, \ldots, \vc{e}_m \in V$ линейно независимы. Другой факт: во всяком $n$-мерном $V$ существуют ортонормированные базисы.

\begin{to_def} 
    Скалярное произведение $(\vc{x} | \vc{e})$, где $\|\vc{e}\| = 1$, называют \textit{проекцией} вектора $\vc{x}$ на прямую $\langle \vc{e} \rangle_{\mathbb{R}}$.   
\end{to_def}

\begin{to_def} 
    Множество всех векторов $\vc{x} \in V$, ортогональных $U \subset V$, есть подпространство $U^{\bot}$, которое называется \textit{ортогональным дополнением к $U$}. 
\end{to_def}

\begin{to_thr}[процесс Грама -- Шмидта]
\textbf{П}усть $\vc{e}_1, \dots, \vc{e}_m$ --  ЛНеЗ система $\subset \vR{m}$. \textbf{Т}огда $\exists$ ортонормированная система векторов $\vc{e}_1', \dots, \vc{e}_m'$ такая, что $L_i = \langle \vc{e}_1, \dots, \vc{e}_i \rangle$ и $L_i' = \langle \vc{e}_1', \dots, \vc{e}_i' \rangle$ совпадают при $i = 1,2, \dots, m \leqslant n$.
\end{to_thr}
  
\begin{proof}[$\triangle$]
    % с. 109
    Пусть построена система для $k$ векторов. Найдём $\vc{e}_{k+1}$. Верно, что $L_{k+1} = \langle \vc{e}_{1}, \ldots, \vc{e}_k, \vc{v}\rangle$, где
    $$
        \vc{v} = \vc{e}_{k+1} - \sum \lambda_i \vc{e}_i'
    $$
    с произвольными $\lambda$. Подберём их так, чтобы $\vc{v} \bot L'_k$. Для этого необходимо и достаточно условий
    $$
        0 = (\vc{v} | \vc{e}_j') = (\vc{e}_{k+1} | \vc{e}_j') - \left(\sum_{i=1}^k \lambda_i \vc{e}_i' | \vc{e}_j \right) = 
        (\vc{e}_{k+1} | \vc{e}_j') - \lambda_j, \hspace{0.5cm} j = 1, \ldots, k.
    $$
    Таким образом, при $\lambda_j = (\vc{e}_{k+1} | \vc{e}_j')$ получаем вектор $\vc{v} \neq \vc{0}$, ортогональный к $L_k'$. Полагая $\vc{e}_{k+1}' = \mu \vc{v}$ придём к ортонормированной системе. 
\end{proof}

Как следствие, всякая ортонормированная система векторов $V$ дополняема до ортонормированного базиса.


\begin{to_thr}
% Кострикин, с. 110.
    \textbf{П}усть $L$ -- подпространство конечномерного евклидова пространства $V$, $L^{\bot}$ -- его ортогональное дополнение. \textbf{Т}огда 
    \begin{equation}
        V = L \oplus L^{\bot}, \hspace{1cm} L^{\bot \bot} = L.
    \end{equation}
\end{to_thr}

\begin{proof}[$\triangle$]
    Возьмем в $L$ какой-нибудь ортонормированный базис $(\vc{e}_{1}, \ldots, \vc{e}_m)$. Пусть $\vc{w} \in V$. Рассмотрим вектор
    $$
        \vc{v} = \vc{w} - \sum_{i=1}^m (\vc{w} | \vc{e}_i) \vc{e}_i.
    $$
    Так как $(\vc{v} | \vc{e}_j) = (\vc{w} \mid \vc{e}_j) - \sum_{i=1}^{m} (\vc{w} | \vc{e}_i) (\vc{e}_i | \vc{e}_j) = (\vc{w} | \vc{e}_j)  - (\vc{w} | \vc{e}_j) \cdot 1 = 0\; \; \forall j \leq n$. Получается $\vc{v}$ ортогонален $L$. Это значит, что $\vc{w} = \vc{u} + \vc{v}$, где $\vc{u} = \sum_{i=1}^{m} (\vc{w} | \vc{e}_i) \vc{e}_i \in L$ и $\vc{v} \in L^{\bot}$. Итак, $V = L + L^{\bot}$. 

    Пусть $\vc{x} \in L \cap L^{\bot}$. Так как $\vc{x} \in L$, то $(\vc{x} | L^{\bot})=0$. Но и $\vc{x} \in L^{\bot}$, так что $(\vc{x} | \vc{x}) = 0$.

    Бонус. Из разложения $\vc{w} = \vc{u} + \vc{v}$ легко получить, что $L^{\bot \bot} = L$. 
\end{proof}

\subsubsection{\xmark Изоморфизмы}

\begin{to_thr}
    Любые евклидовы пространства $V$, $V'$ одинаковой конечной размерности изоморфны. Существует изоморфизма $f \colon V \to V'$, сохраняющий скалярное произведение, т.е.
    \begin{equation}
        \Sp{x}{y} = (f(\vc{x}), f(\vc{y}))'
    \end{equation}
\end{to_thr}

\begin{to_thr}
    Отображение $\Phi \colon \vc{v} \to \Sp{v}{*} \equiv \Phi_{v}$ -- естественный изоморфизм $V$ и $V^{*}$. При этом $\Phi$ ОНБ $V$ отождествляется с \textbf{дуальным} к нему базисом $f_1, \dots, f_n$ пространства $V^*$. 
\end{to_thr}



\subsubsection{\xmark Ортогональные матрицы}

\noindent
Для любой ортогональной матрицы:
\begin{equation}
    A\T \cdot A = E
\end{equation} 

\subsubsection{\xmark Симплектические пространства}


% \subsection*{\xmark Эрмитовы векторные пространства}

% \begin{to_def}[\textbf{Скалярное произведение/Эрмитова форма}]
%     Отображение $V \times V \to \mathbb{C}$:\\
%       \begin{tabular}{rll}
%         i)& $(\vc{x}, \vc{y}) = \overline{(\vc{y}, \vc{x})}$ & $\forall \vc{x}, \vc{y} \in V$; \\
%        ii) & $(\alpha \vc{x} + \beta \vc{y}, \vc{z}) = \alpha \Sp{x}{z} + \beta \Sp{y}{z}$ & $\forall \alpha, \beta \in \mathbb{R}$; \\
%        iii) & $\Sp{x}{x} > 0$ & $\forall \vc{x} \neq 0$. \\
%         \end{tabular}
% \end{to_def}

