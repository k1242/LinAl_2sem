\section{Билинейные и квадратичные форма}

\subsection{Билинейная форма}
\begin{to_def} 
	\textbf{Билинейная форма} на линейном пространстве $V$ -- $b \colon V \times V \to \mathbb{F}$, линейное по $\forall$ аргументу.
\end{to_def}

\begin{to_def} 
	$b \in \mathcal{B} (V)$\footnote{$\mathcal{B}(V)$ --- линейное пространство над $\mathbb{F}$.}, а $(\vc{e}_{1},\ldots,\vc{e}_n)$ -- базис в $V $. \textbf{Матрица билинейной формы}: $B = (b(\vc{e}_i,\vc{e}_j)), \, b \underset{e}{\longleftrightarrow}  B $

	Для $\vc{v},\vc{u} \in V$: $\vc{u} \underset{e}{\longleftrightarrow} x$ и $\vc{v} \underset{e}{\longleftrightarrow} y$ --- $b(\vc{u}, \vc{v}) = x^T B y$. ($B = (b_{i j})$).
\end{to_def}

\begin{to_thr} 
	Пусть $e$ -- базис в $V$ ($\dim V = n$), тогда соответствие $\mathcal{B}(V) \rightarrow M_{n\times n}(\mathbb{F})$ осуществляет изоморфизм линейных пространств. Следствие: $\dim \mathcal{B}(V) = n^2$.
\end{to_thr}
 \begin{proof}[$\triangle$]
 	Инъективность: $b_1 (\vc{u},\vc{v}) = x^T B y = b_2(\vc{u},\vc{v}) \Rightarrow b_1 = b_2$;

 	Сюръективность: определяем $b(\vc{u},\vc{v}) = x^T B y$, тогда $b(\vc{e}_i, \vc{e}_j) = b_{i j}$, значит $b \underset{e}{\longleftrightarrow} B$.
 \end{proof}

\begin{to_thr}
	$b \in \mathcal{B}(V)$, $e$ и $e'$ -- базисы в $V$, $e' = e S$,
	$b\underset{e}{\longleftrightarrow} B$ и $b \underset{e'}{\longleftrightarrow} B'$. Тогда $B' = S^T B S$.
\end{to_thr}

\begin{to_def} 
	Матрицы $B$ и $B' = S^T F S$ с $\det A \neq 0$ --- \textbf{конгруэнтны}. Ранг $B$ в каком-то базисе соответствующей $b$ называется \textbf{рангом} билинейной формы. $\text{rg}b$ инвариантен относительно изменения базиса.
\end{to_def}

\subsection{Симметричные и кососимметричные формы}

\begin{to_def} 
\begin{tabular}{|l|l|}
\hline
 Симметричная билинейная форма. & Кососимметричная билинейная форма.\\
\hline
\hline

 $\forall \vc{u}, \vc{v} \in V: \: b(\vc{u},\vc{v}) = b(\vc{v}, \vc{u})$ & 
$\forall \vc{u}, \vc{v} \in V: \: b(\vc{u},\vc{v}) = - b(\vc{v}, \vc{u}) \overset{}{\Leftarrow} b(\vc{u},\vc{u}) = 0 $\\

&\\

$\mathcal{B}^- (V)$ ---  \textbf{симметричные} формы на $V$ &
$\mathcal{B}^- (V)$ ---  \textbf{кососимметричные} формы на $V$ \\

&\\

$b \underset{e}{\longleftrightarrow} B, \, b \in \mathcal{B}^+ (V) \Leftrightarrow B^T = B$ &
$b \underset{e}{\longleftrightarrow} B, \,  b \in \mathcal{B}^- (V) \Leftrightarrow B^T = -B$ \\

&\\

$b^+(\vc{u},\vc{v}) = \frac{b(\vc{u},\vc{v}) + b (\vc{v}, \vc{u})}{2}$ &
$b^-(\vc{u},\vc{v}) = \frac{b(\vc{u},\vc{v}) - b (\vc{v}, \vc{u})}{2}$ \\

\hline
\end{tabular}
\end{to_def}

\begin{to_thr} 
	Пусть $\text{char}(\mathbb{F}) \neq 2 $, $\mathcal{B} = \mathcal{B}^+ \oplus \mathcal{B}^-$. 
\end{to_thr}

\begin{to_def} 
	Пусть $b \in \mathcal{B}^\pm (V)$. Тогда \textbf{ядром} формы $b$ называется:
	$\Ker b := \left\{ \vc{v} \in V \colon \forall \vc{u} \in V \: b(\vc{u}, \vc{v}) = 0\right\} = \left\{ \vc{u} \in V \colon \forall \vc{v} \in V \: b(\vc{u}, \vc{v}) = 0\right\}$ (соответственно левое и правое ядра). 
\end{to_def}

	% Другое оформление
	% \begin{minipage}{0.45\linewidth}
	% 	Симметричная билинейная форма.

	% 	$\forall \vc{u}, \vc{v} \in V: \: b(\vc{u},\vc{v}) = b(\vc{v}, \vc{u})$

	% 	$\mathcal{B}^- (V)$ ---  \textbf{симметричные} формы на $V$

	% 	$b \underset{e}{\longleftrightarrow} B$, $B^T = B$
	% \end{minipage}
	% \hfill
	% \begin{minipage}{0.45\linewidth}
	% 	Кососимметричная билинейная форма.

	% 	$\forall \vc{u}, \vc{v} \in V: \: b(\vc{u},\vc{v}) = - b(\vc{v}, \vc{u}) \overset{}{\Leftarrow} b(\vc{u},\vc{u}) = 0$

	% 	$\mathcal{B}^- (V)$ ---  \textbf{кососимметричные} формы на $V$

	% 	$b \underset{e}{\longleftrightarrow} B$, $B^T = -B$
	% \end{minipage}


\begin{to_thr} 
	$\dim \Ker b = \dim V - \rg b$.
\end{to_thr}

\begin{proof}[$\triangle$]
	1) Рассмотрим базис $e = (\vc{e},\ldots,\vc{e}_n)$, и $b \underset{e}{\longleftrightarrow} B$. Пусть $\vc{v}\in V, \, \vc{v} \underset{e}{\longleftrightarrow} x, \, \vc{v} \in \Ker b \Leftrightarrow \forall \vc{u} \: b(\vc{u},\vc{v}) = 0$;

	2) Или равносильно: $\forall i \: b(\vc{e}_i, \vc{v}) = 0 \Leftrightarrow E B X = 0 \Leftrightarrow B X = 0$. Пространство решений это ОСЛУ имеет требуемое равенство: $\dim \Ker b = \dim V - \rg B$.
\end{proof}

\subsection{Ортогональные и невырожденные}
\begin{to_def} 
	Пусть $b \in \mathcal{B}^{\pm}(V), \, \vc{u},\vc{v} \in V$. $\vc{u}$ и $\vc{v}$ \textbf{ортогональны} относительно $b$, если $b(\vc{u},\vc{v}) = 0$. 

	Для $U \subseteq V$, \textbf{ортогональное дополнение} $U$ -- $U^\bot \left\{ \vc{v} \in V \colon \forall \vc{u} \in U \: b(\vc{u},\vc{v}) = 0\right\}$.
\end{to_def}

\begin{to_def} 
	Пусть  $b \in \mathcal{B}^\pm (V)$, форма $b$ --- \textbf{невырожденная},
	% \footnote{Униженные и оскорблённые}
	 если $\rg b = \dim V$.
\end{to_def}

\begin{to_thr} 
	$\dim U^\bot \geq \dim V - \dim U$. А если форма $b$ -- невырождена, то $\dim U^\bot = \dim V - \dim U$
	\label{dim_ort}
\end{to_thr}

\begin{proof}[$\triangle$]
	1) Выберем в $V$ базис $(\vc{e}_1,\ldots,\vc{e}_n)$ так, чтобы первые $k$ векторов были базисом $U$.

	2) Тогда, если $\vc{v} \underset{e}{\longleftrightarrow} x$: $\vc{v} \in U^\bot \Leftrightarrow \forall i = 1,\ldots,k: \: b(\vc{e}_i, \vc{v}) = 0 \Leftrightarrow \left(E_k | 0\right) B x = 0$.

	3) ОСЛУ (2) состоит из $k$ строк матрицы $B$, значит её ранг $\leq k \Longrightarrow \dim U^\bot = n -k$.

	4)Если $b$ -- невырождена, то строчки $B$ -- ЛНеЗ $\Longrightarrow$ ОСлу имеет ранг $k \Longrightarrow \dim U^\bot = n -k$.
\end{proof}

\begin{to_def} 
	$b \in \mathcal{B}^\pm (V), \, U \subseteq V$. $U$ --- \textbf{невырожденное} относительно $b$, если $b \big|_U \in \mathcal{B}^\pm (U)$ -- невырождена. 
\end{to_def}

\begin{to_thr} 
	 Пусть $b \in \mathcal{B}^\pm (V), \, U \subseteq V$. Тогда $U$ -- невырождено $<==> V = U \oplus U^\bot$.
\end{to_thr}

\begin{proof}[$\triangle$]
	1) Базис как в теореме \eqref{dim_ort}. Матрица $b \big|_U$ -- это подматрица $B$ стоящая в верхнем левом углу.

	2) так как $U$ -- невырождено: $\rg B_U = k \Rightarrow$ первые $k$ строк $B_U$ ЛНеЗ, значит $\dim U^\bot = n -k$.

	3) Кроме того, $\Ker b \big|_U = 0$ так как $\dim \Ker b \big|_U = k - k = 0$. То есть $\forall \vc{v} \in U, \, \vc{v} \neq 0 => \exists \vc{u} \in U: \: b(\vc{u},\vc{v}) = 0$, что означает, что $U \cap U^\bot = \vc{0}$. \textbf{Итак}, $U + U^\bot = U \oplus U^\bot$ и $\dim(U + U^\bot) = k + (n-k) = n$.
\end{proof}

\subsection{Квадратичные формы}
\begin{to_def} 
	$h \colon V \to \mathbb{F}$ --- \textbf{квадратичная форма}, $h(\vc{v}) = b(\vc{v},\vc{v}) \: \forall \vc{v} \in V$, для некоторой $b \in \mathcal{B}(V)$.
\end{to_def}

\begin{to_thr} 
	Если $\text{char}(\mathbb{F}) \neq 2$. Тогда $\forall h \in \mathcal{Q}(V)\, \exists! b \in \mathcal{B}^+(V): \: h(\vc{v}) = b(\vc{v},\vc{v})$ ($\mathcal{Q} \cong \mathcal{B}^+(V)$).
\end{to_thr}

\begin{proof}[$\triangle$]
	$\encircled{\exists}$. Пусть $b \in \mathcal{B}(V):\: b = b^+ + b^- \leadsto h(\vc{v}) = b(\vc{v},\vc{v}) = b^+(\vc{v},\vc{v}) + \cancel{b^-(\vc{v},\vc{v})}$. $h$ задаётся $b^+$.

	$\encircled{!}$. Пусть $h(\vc{v}) = b(\vc{v},\vc{v}), \, b \in \mathcal{B}^+(V)$. Восстановим $b$ по $h$. Для $\vc{u},\vc{v} \in V$:

	$h(\vc{u} + \vc{v}) = b (\vc{u} + \vc{v}, \vc{u} + \vc{v}) = b(\vc{u},\vc{u}) + b(\vc{v},\vc{v}) + 2 b (\vc{u}, \vc{v}) \leadsto b(\vc{u},\vc{v}) = \big[h(\vc{u}+\vc{v}) - h(\vc{u}) - h(\vc{v})\big]/2$\\
	Полученная симметричная форма --- \textbf{билинейная форма полярная к $h$}.
\end{proof}

Пусть $b \underset{e}{\longleftrightarrow} B, \, \vc{u} \underset{e}{\longleftrightarrow} x, \, \vc{v} \underset{e}{\longleftrightarrow} y$. Имеем $b(\vc{u},\vc{v}) = x^T B y = \sum\limits_{ i=1 }^{ n } \sum\limits_{ j=1 }^{ n } b_{i j} x_i, y_j$. Тогда квадратичная форма $h (\vc{v}) = y^T B y = \sum\limits_{ i=1 }^{ n } \sum\limits_{ j=1 }^{ n } b_{i j}y_i y_j$. Если $b$ -- симметричная, то $b_{ i j} = b_{j i}$, а $h(\vc{v}) = \sum\limits_{ i=1 }^{ n } b_{i i} y_i^2 + 2 \sum\limits_{ i<j }^{ n } b_{i j} y_i y_j$.

Отныне характеристика нашего поля ни в коем виде не \textbf{двойка}.

\begin{to_def} 
	 $h \in \mathcal{Q}(V)$ с \textbf{полярной} $b \in \mathcal{B}(V)$, $e$ -- базис. Тогда матрица $h$ -- это матрица $b$  в базисе $e$.

	 Матрица $h \in \mathcal{Q}(V)$ всегда симметрична. Если $h \underset{e}{\longleftrightarrow} B, \, \vc{v} \underset{e}{\longleftrightarrow} x$, то $h(\vc{v}) = b(\vc{v}, \vc{v}) = x^T B x$.
\end{to_def}

\begin{to_thr} 
	Пусть $h \in \mathcal{Q}(V)$. Тогда $\exists e$ -- базис в $V:$ $h$ в этом базисе имеет диагональную матрицу. 
\end{to_thr}

\begin{proof}[$\triangle$]
	1) Индукция по $n = \dim V$. Для $n = 1$ доказывать нечего. Для $h = 0$ тоже.

	2) $n>1$: $\exists \vc{e}_1:\: h(\vc{e}_1) \neq 0$. Тогда $\langle \vc{e}_1\rangle$ -- невырождена относительно полярной к $h$ -- $b$.

	3) То есть $V = \langle \vc{e}_1\rangle \oplus \langle \vc{e}_1\rangle^\bot$. По индукции, в $U = \langle \vc{e}_1\rangle^\bot$ есть базис $(\vc{e}_2,\ldots,\vc{e}_n)$, где $h \big|_U$ диагональна.

	4) Матрица(3) -- $B'$, тогда $h\underset{e}{\longleftrightarrow} B$ в $(\vc{e}_1,\vc{e}_2,\ldots,\vc{e}_n)$ состоит из $B'$ и $h(\vc{e}_1)$ в верхнем левом углу.
\end{proof}

\begin{to_con}
	Пусть $\mathbb{F} = \mathbb{R} \Rightarrow\forall h \in  \mathcal{Q}(V), \, \exists e \in V:\: h \underset{e}{\longleftrightarrow} B \in e$ -- диагональна c $0,\, \pm 1$ на диагонали.
\end{to_con}

\begin{to_def} 
	Над $\mathbb{R}$ $h \in \mathcal{Q}(V)$.  Базис, в котором $h \underset{e}{\longleftrightarrow} B$ -- диагональна с $0,\, \pm 1$ --- \textbf{нормальный базис}.
	Матрица $B$ --- \textbf{нормальная форма} для $h$.
\end{to_def}

\begin{to_def} 
	 Пусть $h \in \mathcal{Q}(V)$ над $\mathbb{R}$ (далее всегда \textbf{$\mathbb{F}  = \mathbb{R}$}). Тогда $h$ называется:

	  	\textbf{положительно полуопределенной},  если $\forall \vc{v} \in V\: h(\vc{v}) \geq 0$ $\Leftrightarrow$ на диагонали $B$ только $0,\, + 1$.

	  	\textbf{положительно определенной}, если $\forall \vc{o} \neq \vc{v} \in V\: h(\vc{v}) > 0$ $\Leftrightarrow$ на диагонали $B$ только $+ 1$. 

	  	\textbf{отрицательно определенной или полуопределенной}.\\
	  	В этих случаях полярная к $h$ билинейная форма приобретает те же названия.
\end{to_def}

\begin{to_def} 
	Пусть $h \in \mathcal{Q}(V)$. Её \textbf{положительный индекс инерции} $\sigma_+(h)$  --- наибольшая размерность подпространства $U\subseteq V$, на которой $h \big|_U$ -- положительно определена. (Отрицательный индекс инерции так же)
\end{to_def}

\begin{to_thr} 
	$\mathcal{Q}(V) \ni h \underset{e}{\longleftrightarrow} B$ -- её нормальный вид в $e$. Тогда на диагонали $B$ стоит ровно $\sigma_+(h)$ единиц и $\sigma_-(h)$ минус единиц. 
\end{to_thr}

\begin{proof}[$\triangle$]
	1) Пусть $B = 
	\begin{pmatrix}
	E_k&&O\\
	&0_l&\\
	O&&-E_m\\	
	\end{pmatrix}$.
	Тогда, если $U = \langle \vc{e}_1,\ldots,\vc{e}_k\rangle$, то матрица $h \big|_U$ -- единичная, то есть $h \big|_U$ -- положительно определена. Для $W = \langle \vc{e}_{k+1},\ldots,\vc{e}_n\rangle$ получаем $h\big|_U$ -- отрицательно полуопределена.

	2) Пусть $U' \subseteq V: \: h\big|_{U'}$ -- положительно определена $\Rightarrow \forall (\vc{0} \neq) \vc{v} \in U' \cap W: \: 0 < h(\vc{v}) \leq 0$ -- невозможно.
	 \textbf{Итак}, $U' \cap W = 0 \Rightarrow \dim U' \leq \dim V - \dim W = k$, в итоге $\sigma_+(h) = k$.
	 Аналогично $\sigma_-(h) = m$.
\end{proof}

\begin{to_con}[Закон инерции]
	Нормальный вид матрицы $h\in \mathcal{Q}(V)$ определён однозначно с точностью до перестановки элементов диагонали.
\end{to_con}

\begin{to_def} 
	$B$ -- симметричная матрица над $\mathbb{R}$ Она обретает такие же названия, как у квадратичной формы, если она её матрица.

	$B$ -- положительно определена $\Leftrightarrow \exists$ невырожденная $A:\: B = A^T A$
\end{to_def}

\begin{proof}[$\triangle$]
	$\encircled{\Rightarrow}$. У соответствующей $h$ нормальный вид -- $E = S^T B S$. Тогда $B = (S^T)^{-1} S^{-1} = (S^{-1})^{T} S^{-1}$.

	$\encircled{\Leftarrow}$. Если $B = A^T A \Rightarrow \forall(\vc{0} \neq)\vc{v} \in V, \, \vc{v} \underset{e}{\longleftrightarrow} x, \: h(\vc{v}) = x^T B x = x^T A^T A x = (A x)^T A x > 0$.

	\textit{Note}: и также для $B$ -- полуопределенной $\Leftrightarrow \exists A:\: B = A^T A$.
\end{proof} 

\begin{to_def} 
	$B$ -- симметричная матрица. Её \textbf{главный минор} $i$-порядка $\Delta_i(B)$ --- это определитель матрицы $i \times i$ в левом верхнем углу. 
\end{to_def}

\begin{to_thr}[Метод Якоби]
	 $\mathcal{Q}(V) \ni h \underset{e}{\longleftrightarrow} B$, причём $\Delta_i (B) \neq 0, \, i = 1,\ldots,n(=\dim V)$. Тогда $\exists e' = e S$, где $S$ -- верхнетреугольная матрица с единицами на диагонали такой, \\что $h \underset{e'}{\longleftrightarrow} 
	 \begin{pmatrix}
	 d_1&&O\\
	 &\ddots&\\
	 O&&d_n\\	
	 \end{pmatrix}$, где $d_i = \frac{\Delta_i(B)}{\Delta_{i-1}(B)}$ ($\Delta_0(B) = 1$).
\end{to_thr}

\begin{proof}[$\triangle$]
	1) Индукция по $n$. При $n =1$ доказывать нечего. Для $n >1$ имеем $\langle \vc{e}_1,\ldots,e_{n-1}\rangle$ - невырожденный относительно билинейной формы, полярной к $h$ (так как $\Delta_{n-1}(B) \neq 0)$).

	2) Значит $V = U \oplus U^\bot$. Разложим $\vc{e}_n = \vc{u} + \vc{e}_n'$ ($u \in U,\, \vc{o} \neq \vc{e}_n' \in U^\bot$). 
	Тогда по предположению индукции: найдётся замена базиса в $U$: $(\vc{e}_1',\ldots,\vc{e}_{n-1}') = (\vc{e}_1,\ldots,\vc{e}_{n-1}) S$, приводящая $h \big|_U$ к диагональному.

	3) Тогда $\vc{e}_n' \in \langle \vc{e}_1',\ldots, \vc{e}_{n-1}'\rangle^\bot = U^\bot$. Получаем $h \underset{e'}{\longleftrightarrow} B' = 
	\begin{pmatrix}[c|c]
		\begin{matrix}d_1&&\\&\ddots&\\&&d_{n-1}\\ \end{matrix} & O\\
		\hline
		O & d_n\\
	\end{pmatrix}$
	$S = \begin{pmatrix}[c|c]
		S'&O\\
		\hline
		O&1\\
	\end{pmatrix}$.

	% $\boxed{\begin{matrix}d_1&&\\&\ddots&\\&&d_{n-1}\\ \end{matrix}}$

	4) Доказав переход индукции осталось вычислить $d_i$. Заметим, что: $\vc{e}_i' \in \langle \vc{e}_1,\ldots,\vc{e}_i\rangle(=\langle \vc{e}_1',\ldots,\vc{e}_i'\rangle)$. 

	5) Пусть $B_i(')$ -- подматрица в $B(')$ в левом верхнем углу ($\Delta_i (B) = |B_i|$). Тогда $B_i' = S_i^T B_i S_i$ ($e_i' = e_i S_i$ и $S_i$ -- верхнетреугольная с единицами на диагонали).

	6) Значит $\Delta_i(B') = |B_i'| = |S_i^T B_i S_i| = |B_i| |S_i|^2 = |B_i| = \Delta_i(B) (= d_1 \ldots d_i) \leadsto$ $d_i = \frac{|B_i'|}{|B_{i-1}'|} = \frac{\Delta_i (B)}{\Delta_{i-1}(B)}$.
\end{proof}

\begin{to_thr}[Критерий Сильвестра]
	 $\mathcal{Q}(V) \ni h \underset{e}{\longleftrightarrow} B$.

	 $h$ -- положительно определена $\Longleftrightarrow \forall i = 1,\ldots,n(=\dim(V))\: \Delta_i(B) > 0$.
\end{to_thr}

\begin{proof}[$\triangle$]
	$\encircled{\Rightarrow}$. $h$ -- положительно определена $\Leftrightarrow B = A^T A$, $A$ -- невырождена. Тогда $\Delta_n (B) = |B| = |A|^2 >0$.

	$\encircled{\Leftarrow}$. Из метода Якоби, на диагонали: $\Delta_1, \, \Delta_2/\Delta_1 \ldots$ все $>0$ $\Rightarrow h$ -- положительно определена. 
\end{proof}

\begin{to_con}
	Если $\forall i = 1, \ldots, n:\:\Delta_i(B) \neq 0 \Rightarrow \sigma_-(h)$ -- число перемен знака в $1, \Delta_1(B),\ldots, \Delta_n(B)$. 
\end{to_con}

\subsection{Кососимметричные и полуторолинейные формы}

\begin{to_thr} 
	Теперь и далее $\mathbb{F}$ -- любое. Пусть $b \in \mathcal{B}^-(V)$. Тогда в $V \, \exists e$, в котором $b \underset{e}{\longleftrightarrow} \begin{pmatrix}
		$\boxed{\begin{matrix}0&1\\-1&0\\\end{matrix}}$&&O\\
		&\ddots&\\
		O&&\boxed{0}\\
	
	\end{pmatrix}$ 
	% $\boxed{\begin{matrix}0&1\\-1&0\\\end{matrix}}$
\end{to_thr}

\begin{proof}[$\triangle$]
	1) Индукция по $n = \dim V$. Если $\rg b = 0$, то доказывать нечего.

	2)Иначе $\exists \vc{e}_1, \vc{e}_2:\: b(\vc{e}_1, \vc{e}_2) \neq0 \Rightarrow \vc{e}_1, \vc{e}_2$ -- ЛНеЗ и, можно считать, $b(\vc{e}_1,\vc{e}_2)= 1$.

	3) Далее, $U = \langle \vc{e}_1, \vc{e}_2\rangle$ -- невырождено относительно $b$
	и $b \big|_U \underset{(\vc{e}_1, \vc{e}_2)}{\longleftrightarrow} B = \begin{pmatrix} 0&1\\ -1&0\\ \end{pmatrix}$.

	4) Значит, $V = U \oplus U^\bot$. По предположению индукции к $b \big|_U^\bot$ получаем $B_1$ в базисе $(\vc{e}_3,\ldots,\vc{e}_n)$.

	5) Тогда, для $e = (\vc{e}_1,\ldots,\vc{e}_n)$ имеем: $b \underset{e}{\longleftrightarrow} 
	\begin{pmatrix}[c|c]
		\begin{matrix}0&1\\-1&0\\\end{matrix} & O\\
		\hline
		O& B_1\\
	\end{pmatrix}
	$.
\end{proof}

Пусть отныне $\mathbb{F} = \mathbb{C}$, $V$ -- линейное пространство над $\mathbb{F}$.

\begin{to_def} 
	$b \colon V \times V \to \mathbb{F}$ --- \textbf{полуторолинейная форма}, если она:
	\begin{equation*}
		\begin{aligned}
			&\text{1) Линейная по первому аргументу:} 
			&\left\{
			\begin{aligned}
				b(\vc{u}_1 +\vc{u}_2, \vc{v}) = b(\vc{u}_1, \vc{v}) + (\vc{u}_2, \vc{v}) \\
				\forall \lambda \in \mathbb{C}:\: b(\lambda \vc{u}, \vc{v}) = \lambda b(\vc{u},\vc{v})	
			\end{aligned}
			\right. \\
			&\text{2) Сопряженно линейная по второму аргументу:}
			&\left\{
			\begin{aligned}
				b(\vc{u},\vc{v}_1 +\vc{v}_2) = b(\vc{u},\vc{v}_1) + (\vc{u},\vc{v}_2) \\
				\forall \lambda \in \mathbb{C}:\: b(\vc{u},\lambda \vc{v}) = \overline{\lambda} b(\vc{u},\vc{v})	
			\end{aligned}
			\right.
		\end{aligned}
	\end{equation*}
\end{to_def}

\begin{to_def} 
	\textbf{Матрица полуторолинейной формы} в базисе $(\vc{e}_1,\ldots,\vc{e}_n)$ --- это $b \underset{e}{<-->} B = (b(\vc{e}_i, \vc{e}_j))$. 

	Если $\vc{u} \underset{e}{\longleftrightarrow} x, \, \vc{v} \underset{e}{\longleftrightarrow} y$, то $b(\vc{u},\vc{v}) = x^T B \overline{y}$.
\end{to_def}

\begin{to_thr} 
	Пространство полуторолинейных форм $S(V) \cong M_{n\times n}(\mathbb{C)})$. Переход: $e' = e S, \, B' = S^T B \overline{S}$.
\end{to_thr}

\begin{proof}
	$|B'| = |S^T| \cdot |B| \cdot |\overline{S}| = |B| \cdot |\det S|^2$
\end{proof}

\begin{to_con}
	$\rg B(=\rg b)$ и $\arg \det B$ не зависят от выбора базиса. 
\end{to_con}

\begin{to_def} 
	Пусть $b \in S(V)$. $b$ называется \textbf{эрмитовой формой}, если $b(\vc{u},\vc{v}) = \overline{b(\vc{v},\vc{u})}$.   
\end{to_def}

\begin{to_lem}
	Пусть $S(V) \ni b \underset{e}{\longleftrightarrow} B$. Тогда $b$ -- эрмитова $\Longleftrightarrow B^T = \overline{B}$.
\end{to_lem}

\begin{to_def} 
	$B \in M_{n \times n}(\mathbb{C})$ --- \textbf{эрмитова}, если $B = \overline{B^T}$. $B^* = \overline{B^T}$ --- \textbf{эрмитово сопряженной } к $B$.
\end{to_def}

\begin{to_def}
	Пусть $b$ -- эрмитова форма на $V$. Тогда $h \colon V \to \mathbb{C}, \, h(\vc{v}) = b(\vc{v},\vc{v})$ --- \textbf{эрмитова квадратичная форма} соответствующая $b$. ($b$ полярна к $h$)
\end{to_def}
	
\begin{to_lem}
	Если $b$ -- эрмитова форм, то 1) $b (\vc{v},\vc{v}) \in \mathbb{R}$; 2) $b \underset{e}{\longleftrightarrow} B:\, |B| \in \mathbb{R}$.	Следствие: $h$ принимает значения лишь из $\mathbb{R}$.
\end{to_lem}

\begin{to_lem}
	Если $b_1 \neq b_2$ -- эрмитовы формы, то соответствующие $h_1 \neq h_2$.
\end{to_lem}

\begin{proof}[$\triangle$]
	Восстановим $b$ по $h$: $h(\vc{u} + \vc{v}) = h(\vc{u}) + h(\vc{v}) + b(\vc{u}, \vc{v}) + b(\vc{v}, \vc{u}) = h(\vc{u}) + h(\vc{v}) + 2\Re (b(\vc{u},\vc{v})) \Rightarrow$

	$\Re(b(\vc{u}, \vc{v})) = [h(\vc{u}, + \vc{v}) - h(\vc{u}) - h(\vc{v})]/2,$

	$b(\vc{u},i \vc{v}) = - i b(\vc{u}, \vc{v}) \Rightarrow\Im (b(\vc{u},\vc{v})) = \Re (-i b(\vc{u},\vc{v})) = \Re (b(\vc{u}, i \vc{v})) = [h(\vc{u}, + i \vc{v}) - h(\vc{u}) - h(i \vc{v})]/2$

	Следствие: Соответствие между эрмитовыми и квадратичными эрмитовыми формами биеткивно и $\mathbb{R}$-линейно 
\end{proof}

Значит линейные вещественные пространства эрмитовых и эрмитовых квадратичных форм изоморфны.

Пусть $b$ -- эрмитова форма. $\Ker b = \{\vc{u} \colon \forall \vc{v} \in V\, b(\vc{u},\vc{v}) = 0\} = \{\vc{v}\colon \forall \vc{u} \in V\, b(\vc{u},\vc{v})=0\}$.

$\dim U^\bot \geq \dim V - \dim U$, если $b$ -- невырождена $\Longrightarrow \dim U^\bot = \dim V - \dim U$.

$V = U \oplus U^\bot \Longleftrightarrow U$ -- невырождено относительно $b$ (то есть $b\big|_U$ -- невырождена).

\begin{to_thr} 
	Пусть $h$ -- эрмитова квадратичная форма, тогда $\exists e:\: h \underset{e}{\longleftrightarrow} B$ -- диагональна с $\{0, \pm 1\}$	 
\end{to_thr}

\begin{proof}[$\triangle$]
	1)Приведём к диагональному виду индукцией: $h(\vc{e}_1) \neq 0:\: \langle \vc{e}_1\rangle$ -- невырождена относительно $b \Longrightarrow$

	2) $V = \langle \vc{e}_1\rangle \oplus \langle \vc{e}_1\rangle^\bot$ применим индукцию: $h \underset{e}{\longleftrightarrow}\text{diag}(\alpha_i)$.

	3) Нормируем векторы: $\vc{e}_i = \frac{\vc{e}_i}{\sqrt{|h (\vc{e}_i)|}}$, если $h(\vc{e}_1) \neq 0$.
\end{proof}

\begin{to_def} 
	Пусть $h$ -- эрмитова квадратичная форма. $h$ --- \textbf{положительно (полу)определена}, если $\forall v:\: h(\vc{v}) >0 (\geq)$. Аналогично с \textbf{отрицательной} (полу)определенностью.  
\end{to_def}

\begin{to_def} 
	\textbf{Положительный/отрицательный индекс инерции} $\sigma_+(h), \sigma_-(h)$ --- как и раньше. 

	\textbf{Закон инерции}: В нормальном виде формы $b$ ровно $\sigma_+(h)$ единиц и $\sigma_-(h)$ минус единиц.
\end{to_def}

\begin{to_thr} [Метод Якоби и Критерий Сильвестра]
	 АНАЛОГИЧНО
\end{to_thr}