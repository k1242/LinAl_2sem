\section{Начала тензорного исчисления}

\begin{to_def}[понятие тензора]
    Пусть $\F$ -- поле, $V(\F)$ - векторное пространство, $V^*$ -- сопряженное к $V$, $p$ и $q$ -- целые числа $\geqslant 0$. Всякое $(p+q)$-линейное отображение 
    \begin{equation}
        f \colon V^p \times (V^*)^q \to \F
    \end{equation}
    называется \textbf{тензором на $V$ типа $(p, q)$} и валентности (или ранга) $p+q$.
\end{to_def}

\marginpar{\small $\otimes$}
\begin{to_def}
    Пусть $f \colon V_1 \times \dots \times V_r \to \F$, $g \colon W_1 \times \dots \times W_s \to \F$.
    Под тензорным произведением $f$ и $g$ понимают отображение
    \begin{equation}
        f \otimes g \colon V_1 \times \dots \times V_r \times W_1 \times \dots \times W_s \to \F,
    \end{equation}
    определенное формулой
    \begin{equation}
        (f \otimes g)(\vc{v}_1, \dots, \vc{v}_r; \vc{w}_1, \dots, \vc{w}_s) = f(\vc{v}_1, \dots, \vc{v}_r) \cdot g(\vc{w}_1, \dots, \vc{w}_s)
    \end{equation}
\end{to_def}


\begin{figure}[ht!]
\begin{minipage}{0.5\textwidth}
\center
        \begin{tabular}{c|lll}
            \toprule
            $p/q$   &0          &1                          & 2    \\
            \midrule
            0       & $\const$  & $f$        & $b(\vc{x}, \vc{y})$       \\
            1       & $\vc{x}$  &  $L \in \mathcal{L}(V)$   &       \\  
            2       & $b^*(f, g)$     &&       \\
            \bottomrule
        \end{tabular}
\end{minipage}
\begin{minipage}[b]{0.3\textwidth}
        \begin{tabular}{rll}
        1) & $\otimes \colon \mathbb{T}^{p}_{q} \times \mathbb{T}^{p'}_{q'} \to \mathbb{T}^{p+p'}_{q+q'}$; &\\
        2) & ассоцитивность     & \checkmark    \\
            & дистрибутивность  & \checkmark \\
            & \cancel{комутативность}    & \xmark       \\
        \end{tabular}
\end{minipage}
\end{figure}

\noindent
\socrat. Базис в $V$ и $V^*$ выбирается:
\begin{equation}
\begin{aligned}
    (\vc{e}_i, e^j) = \delta_i^j = 
        \left\{
            \begin{aligned}
                0, \text{ если }  i \neq j, \\
                1,  \text{ если }  i = j,   \\
            \end{aligned} 
        \right. \\
    f(\vc{x}) = (f, \vc{x}) = \sum\nolimits_i \alpha^i  \beta_i = \alpha^i  \beta_i.
\end{aligned}
\end{equation}

\begin{to_def}[компоненты тензора]
    Значения тензора обозначаются в виде:
    \begin{equation}
            T_{i_1, \dots, i_p}^{j_1, \dots, j_p} := T(\vc{e}_{i_1}, \dots, \vc{e}_{i_p}, e^{j_1}, \dots, e^{j_q}).
    \end{equation}  
    Числа  $\numt$ называются \textbf{координатами} тензора $T$ в базисе $(\vc{e}_1, \dots, \vc{e}_n)$
\end{to_def}

\begin{proof}[$\triangle$]
Достаточно построить \textit{разложимый} тензор
\begin{equation*}
    T_1(\vc{e}_{i_1}, \dots, \vc{e}_{i_p}, e^{j_1}, \dots, e^{j_p}) = \numt,
\end{equation*}
и воспользоваться тем, что тензор $T$ полностью определяется своими координатами. 

Далее остается показать, что разложимые тензоры, отвечающие различным наборам индексов линейно независимы.
\end{proof}

\begin{to_thr}
    Тензоры типа $(p, q)$ на $V$ составляют $\ten(V)_p^q$ размерности $n^{p+q}$ с базисными векторами
    \begin{equation}
        e^{i_1} \otimes \dots \otimes  e^{i_p} \otimes \vc{e}_{j_1} \otimes \dots \otimes \vc{e}_{j_q},
    \end{equation}
    При том $\exists ! T$ с координатами $\numt$.
\end{to_thr}

\begin{to_thr}
    При переходе от дуальных базисов $(\vc{e}_i)$, $(e^i)$ пространств $V$ и $V^*$ к новым дуальным базисам тех же пространств:
    \begin{equation}
        \vc{e}_k' = a^i_k \vc{e}_i, \hspace{1cm} e^{'k} = b^k_i e^i, \text{ где } (a_{ij})^{-1}) = (b_{ij}), 
    \end{equation}
    координаты тензора $T$ преобразуются по формулам
    \begin{equation}
        \numt = \sum_{i', j'} 
        b_{i_1', \dots, i_p'}^{i_1, \dots, i_p}
        \cdot 
        T'\phantom{|}_{i_1, \dots, i_p}^{j_1, \dots, j_p} 
        \cdot
        a_{j_1' \dots j_p'}^{j_1 \dots j_p}
    \end{equation}
\end{to_thr}

\noindent
\textbf{\# тензорное произведение пространств}

\noindent
\textbf{\# тензорное произведение операторов}

\phantom{42} \marginpar{\textit{\small свёртка}}

\begin{to_def}[свёртка]
    Зафиксировав все переменные кроме $\vc{x}_r$ и $u_s$, получим билинейную форму:
    \begin{equation}
        f(\vc{x}_r, u_s) := T(\dots, \vc{x}_r, \dots, u_s, \dots).
    \end{equation}
    Тогда \textbf{инвариантная} сумма вида
    $
        \overline{T} = f(\vc{e}_k, e^k)
    $
    называется \textbf{свёрткой тензора} $T$ по $r$-му ковариантному и $s$-му контрвариантному индексу.
\end{to_def} 

\noindent
\socrat. Если обозначить свёртку по индексам $r$, $s$ символом $\tr^s_r$, то $\tr^s_r$ -- линейное отображение:
\begin{equation}
    \tr^s_r: \ten^q_p(V) \to \ten_{p-1}^{q-1}(V).
\end{equation}

\begin{to_thr}
    Свёртка вида $\tr^s_r$ тензора $T \in \ten^q_p$ является тензор $\overline{T} \in \ten^{q-1}_{p-1}$ с координатами
    \begin{equation}
        \overline{T}^{
        j_1, \dots, j_{s-1}, j_{s+1}, \dots, j_q
        }_{
        i_1, \dots, i_{r-1}, i_{r+1}, \dots, i_p
        } = 
        \sum_k T^{
        j_1, \dots, j_{s-1}, k, j_{s+1}, \dots, j_q
        }_{
        i_1, \dots, i_{r-1}, k, i_{r+1}, \dots, i_p
        }
    \end{equation}
\end{to_thr}

\marginpar{$S\ten^p(V)$}
Для любой перестановки $\pi \in S_p$ положим 
\begin{equation}
    f_{\pi}(T)(\vc{x}_1, \dots, \vc{x}_p) = T(\vc{x}_{\pi(1)}, \dots, \vc{x}_{\pi(p)})
\end{equation}
\begin{to_def}
    Тензор $T$ типа $(p, 0)$ называется \textbf{симметричным}, если $\forall \pi \in S_p$ $f_{\pi}(T) = T$. \textbf{Симметризацией} $T \in \ten^0_p (V)$ называется отображение
    \begin{equation}
        S(T) = \frac{1}{p!} \sum_{\pi \in S_p} f_{\pi} (T) \colon  \ten^0_p (V) \to \ten^0_p (V).
    \end{equation}
\end{to_def}

\noindent
\socrat. Подпростраснство сим. тензоров типа $\ten^0_p (V)$ обозначим $\ten^+_p (V)$.

\noindent
\socrat. Действие $S$: 1) $S^2 = S$, $\Im S = \ten^+_p (V)$.

\noindent
\socrat\footnote{
    стр. 281, Кострикин.
}
. Пространства $\F[X_1, \dots, X_n]_p$ и $\ten^+_p (V)$ биективны. Тогда
\begin{equation}
    \dim \F[X_1, \dots, X_n]_p = \dim \ten^+_p (V) = 
    \begin{pmatrix}
        n+p-1 \\ p
    \end{pmatrix}
\end{equation}

\begin{to_def}
    Ассоциативная и комутативная \textbf{симметрическая алгебра} пространства $V$:
    \begin{equation}
        S(V) = \oplus_{p=0}^{\infty} S\ten^p (V),
    \end{equation}
    где $\vee$ выступает в качестве умножения.

\end{to_def}

\marginpar{$\Lambda^p(V)$}

\begin{to_def}
    Назовём тензор $T$ кососимметричным, если 
    \begin{equation}
        f_{\pi}(T) = \sign(\pi) \cdot T \hspace{1cm} \forall \pi \in S_p.
    \end{equation}
\end{to_def}

\begin{to_def}
    \textbf{Альтернрованием} называется отображение
    \begin{equation}
        A(T) = \frac{1}{p!} \sum_{\pi \in S_p} \sign(\pi) \cdot f_{\pi} (T) \colon  \ten^0_p (V) \to \ten^0_p (V).
    \end{equation}
\end{to_def}

\noindent
\socrat. Действие $A$: 1) $A^2 = A$, 2) $\Im A = \Lambda^+_p (V)$, 3) $A(f_{\sigma}(T)) = \sign(\sigma) A(T)$.

\marginpar{\small $\wedge$}

\begin{to_def}
    Зададим \textbf{операцию внешнего умножения}
    \begin{equation}
        \wedge \colon \Lambda(V) \times \Lambda(V) \to \Lambda (V),
    \end{equation}
    полагая 
    $
        Q \wedge R = A(Q \otimes R)
    $
    для любого $q$-вектора $Q$ и любого $r$-вектора $R$. 
\end{to_def}

\begin{to_def}[алгебра Грассмана]
    Алгебра $\Lambda (V)$ над $\F$ называется \textbf{внешней алгеброй} пространства $V$:
    \begin{equation}
        \Lambda(V) = \oplus_p^n \Lambda^p (V)
    \end{equation}
\end{to_def}

\begin{to_thr}
    Внешняя алгебра ассоциативна.
\end{to_thr}

\noindent
$\boxed{\Rightarrow}$. Пусть $\vc{x}_1, \vc{x}_2, \dots, \vc{x}_p$ -- произвольные векторы из $V$. Тогда\footnote{
    с. 287, Кострикин.
} 
\begin{equation}
    \vc{x}_1 \wedge \vc{x}_2 \wedge \dots \wedge \vc{x}_p = A(\vc{x}_1 \otimes \vc{x}_2 \otimes \dots \otimes \vc{x}_p).
\end{equation}

\begin{to_thr}
    Пусть $(\vc{e}_1, \dots, \vc{e}_p)$ -- базис $V$. Тогда 
    \begin{equation}
        \vc{e}_1 \wedge \vc{e}_2 \wedge \dots \wedge \vc{e}_p, \hspace{1cm} 1 \leqslant i_1 < \dots < i_p \leqslant n
    \end{equation}
    образуют базис пространства $\Lambda^p(V)$.
\end{to_thr}

\noindent
$\boxed{\Rightarrow}$. Внешняя алгебра $\Lambda(V)$ пространства $V$ имеет размерность $2^n$. При этом 
\begin{equation}
    \dim \Lambda^p (V) = \begin{pmatrix}
        n \\ p
    \end{pmatrix}.
\end{equation}
Базис пространства $\Lambda^n (V)$ состоит из одного $n-$вектора $$\vc{e}_1 \wedge \vc{e}_2 \wedge \dots \wedge \vc{e}_p$$.

\noindent
\socrat. Внешняя алгебра $V$ \textbf{антикоммутативна}:
\begin{equation}
    Q \in \Lambda^q(V), R \in \Lambda^r(V) \Rightarrow Q \wedge R = (-1)^{qr} R \wedge Q.
\end{equation}


\noindent
\# связь с определителями\\
\# векторные подпространства и $p$-векторы\\
\# условия разложимости $p$-векторов
