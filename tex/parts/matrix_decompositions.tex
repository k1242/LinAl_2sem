\subsubsection{Нормальные операторы}
Эрмитовы и унитарные операторы входят в естественный, более широкий класс, диагонализируемых операторов.

\begin{to_def} 
    Пусть $V$ -- эрмитово пространство. Линейный оператор $\A \colon V \to V$, обладающий свойством 
    $$
        \A \cdot \A^* = \A^* \cdot \A,
    $$
    называется \textit{нормальным}. Его матрица в любом базисе также называется нормальной.
\end{to_def}

Оператор $\A$ нормален вместе с $\A - \lambda \mathcal E$. Из нормальности $\A$ вытекает, что 
$$
    \|\mathcal A \vc{x} \|^2 = \|\A^* \vc{x}\|^2.
$$
Заменяя $\A$ на $\A - \lambda \mathcal E$, получаем, что
\begin{equation}
\label{13eq}
    \A \vc{x} = \lambda \vc{x} \; \Longleftrightarrow \; \A^* = \overline{\lambda} \vc{x}
\end{equation}



Определение нормального оператора переностися на бесконечномерные гильбертовы пространства и находит там многочисленны применения. Постараемся опичать класс диагонализируемых операторов на эрмитовом пространстве.

\begin{to_thr} 
    Эквивалентны следующие условия: 

    а) $\A \colon V \to V$ -- оператор, диагонализируемый в ортонормированном базисе пространства $V$;

    б) $\A$ -- нормальный оператор.
\end{to_thr}

\begin{proof}[$\triangle$]
    Докажем, что б) $\Longrightarrow$ а). Выберем собственное значение $\lambda$ оператора $\A$ и положим $V^{\lambda} = \{ \vc{x} \in V \mid \A \vc{x} = \lambda \vc{x}\}$. Из \eqref{13eq} следует, что 
    $$
        \A^* (V^{\lambda}) \subseteq V^{\lambda}.
    $$
    Так как $(\A^*)^* = \A$, то изсимметрии подпространство $(V^\lambda)^{\bot}$ также $\A^*$-инвариантно. Ограничения операторов коммутируют. Применяя индукцию по размерности $n = \dim V$, мы можем считать, что $\A$  диагонализируется. Для $V^{\lambda}$ это верно по определению, а поскольку $V = V^\lambda \oplus \left(V^\lambda\right)^{\bot}$, то всё доказано.  
\end{proof}


\begin{to_thr} 
    Каждому нормалному оператору $\A$ на конечномерном пространств $V$ отвечают попарно различные числа $\lambda_{1}, \ldots, \lambda_m, \; 1 \leq m \leq n = \dim V$, и взаимно ортогональные проекторы $\mathcal P_{1}, \ldots, \mathcal P_m$, отличные от $\mathcal O$ и такие, что:

    а) $\sum_j \mathcal P_j = \mathcal E$;
    б) $\sum_j \lambda_j \mathcal P_j = \mathcal A$ -- спектральное разложение оператора $\A$,  так что $\lambda_j \in \text{Spec} (\A)$;
    в) разложение б) единственно;
    г) сущесвтуют многочлены $f_{1}(t), \ldots, f_m(t)$, такие, что
    $f_i (\lambda_j) = \delta_{ij}$ и $f_i(\mathcal A) = \mathcal P_i$.
\end{to_thr}

\begin{proof}[$\triangle$]
    потом :(
\end{proof}

\begin{to_lem} 
    Пусть $\mathcal A, \mathcal B$ -- коммутирующие операторы на $V^{\mathbb{ C}}$. Тогда $\mathcal A$ и $\mathcal B$ имеют общий собственный вектор. 
\end{to_lem}


\begin{proof}[$\triangle$]
   ... 
\end{proof}


\begin{to_thr} 
    Два эрмитовых оператора $\mathcal A, \mathcal B$ или две изометрии $\mathcal A, \mathcal B$ на $n$-мерном эрмитовом пространстве $V$ одновременно приводятся к диагональному виду в некотором ортонормированном базисе тогда, и только тогда, когда они перестановочны.
\end{to_thr}

\begin{proof}[$\triangle$]
    Из диагонализируемости в некотором базисе следует перестановочность операторов.

    Обратно, пусть $\mathcal{AB} = \mathcal{BA}$. Тогда у них есть общий собственный вектор $\vc{e}_{1}$. Подпространство $W = \langle \vc{e}_1\rangle^{\bot}$ размерности $n-1$ инвариантно относительно $\mathcal A$ и отсносительно $\mathcal B$ в силу их эрмитовости или унитарности. Ограничения $\mathcal A$ и $\mathcal B$  на $W$ будут перестановочными эрмитовыми (соответсвенно унитарными) операторами. Индукция по размерности приводит к явной конструкции ОНБ, в котором $\mathcal A$ и $\mathcal B$ запишутся в диагональной форме.
\end{proof}

Стоит заметить, что перестановочность эрмитовых операторов $\mathcal A, \mathcal B$ эквивалентна эрмитовости оператора $\mathcal{AB}$. 


\subsubsection{Положительно определенные операторы}
\begin{to_def} 
    Эрмитов (или линейный симметричный) оператор $\A$ называется \textit{положительно определенным}, если $(\A \vc{x} | \vc{x}) > 0$ для $\forall x\vc{x} \neq \vc{0}$ из $V$. 
\end{to_def}


\begin{to_thr} 
    Пусть $V$ -- пространство со скалярном произведением $(*|*)$. Следующие свойства линейных операторов на $V$ эквиваентны:

    1) $\mathcal A = \mathcal B^2$, $\mathcal B^* = \mathcal B$;

    2) $\mathcal A = \mathcal C \mathcal C^*$;

    3) $(\mathcal A \vc{x} | \vc{x}) \geq 0.$
\end{to_thr}


\subsubsection{Полярное разложение}

Параллелизм между линейными операторами и комплексными числами простирается дальше, вплоть до $z = |z| e^{i \varphi} = \sqrt{z \overline{z}} e^{i \varphi}$.

\begin{to_thr} 
% что за леммы 5 и 6?
    Всякий невырожденный линейный оператор $\mathcal A$ на эрмитовом (или евклидовом) векторном пространстве $V$ может быть представлен в виде 
    \begin{equation}
    \label{qp}
        \mathcal A = \mathcal P \mathcal Q,
    \end{equation}
    где $\mathcal P$ -- положительно определенный оператор, а $\mathcal Q$ -- изометрия (унитарный или ортогональный оператор). Разложение единственно и называется \textit{полярным разложением} оператора $\mathcal A.$ 
\end{to_thr}


\begin{proof}[$\triangle$]
% добавить сюда предложения 1 и предложение 2
    Во-первых $\mathcal A \mathcal A^* = \mathcal P^2$, где $\mathcal P$ -- положительно определенный оператор, являющийся единственным квадратным корнем из $\A \A^*$. Очевидно, что $\mathcal P$ -- обратим. Положим $\mathcal Q = \mathcal P^{-1} \mathcal A$, получим выражение 
\end{proof}

