\subsection*{Квадрики}

\subsubsection{Квадратичная функция в афинном пространстве}

\begin{to_def} 
    Положим $Q(\dot 0 + \vc{x}) = q(\vc{x}) = 2l(\vc{x}) + \varphi_0$ 
\end{to_def}

\begin{to_def} 
    Точку $\dot p \in A$ назовём \textit{центром} (или \textit{центральной точкой})  для квадратичной функции $Q$, если
    $$
        Q(\dot p + \vc{y}) = Q(\dot p) + q(\vc{y}) \hspace{0.5cm}  \forall \vc{y} \in V.
    $$
\end{to_def}

\begin{to_thr} 
% стр. 220
    Пусть $Q$ -- квадратичная функция ранга $r$ на $n$-мерном аффинном пространве $\mathbb{A}$ над $\mathbb{F}$. Если множество $C(Q)$ пусто и, значит $r < n$, то путем надлежащего выбора системы координат $\{\dot o; \vc{e}_{1}, \ldots,  \vc{e}_n\}$ функция $Q$ приводистя к виду
    $$
        Q(\dot 0 + \vc{x}) = \alpha_{1} x_{1}^2 + \ldots  + \alpha_r \vc{x}_r^2 + 2x_{r+1}
    $$
    с ненулевыми скалярами $\alpha_1, \ldots, \alpha_r$; в этом случае $\Ker q$ есть подпространство решений системы $x_{1} = \ldots = x_r = 0$, где $q$ -- квадратичная форма, связанная с $Q$. 

    Если $Q$ центральна, то выбором надлежащей системы координат с началом в центральной точке $\dot o$ её можно привести к виду
    $$
        Q(\dot o + \vc{x}) = \vc{a}_{1} x_1^2 + \ldots  + \alpha_r x^2_r + \varphi_0;
    $$
    в этом случае $Q(]dot o') = \varphi_0 \; \; \forall \dot o' \in C(Q)$. Вышеупомянутые функции аффинно неэквивалентны.
    \end{to_thr}


Как следствие, над $\mathbb{R}$ всякая квадратичная функция $Q$ может быть приведена, причём единственым образом, к одному из канонических видов:
\begin{align}
    Q(\dot o + \vc{x}) = x_1^2 + \ldots + x_s^2 - x_{s+1}^2 - \ldots  - x_r^2 + 2x_{r+1};
    Q(\dot o + \vc{x}) = x_1^2 + \ldots + x_s^2 - x_{s+1}^2 - \ldots  - x_r^2 + \varphi_0.
\end{align}

Или, что эквивалентно, \textit{две квадратичные функции аффинно эквиваленты тогда, и только тогда, когда, когда они имеют одинковые ранги и одинаковые сигнатуры и когда они обе либо нецентральн, либо центральны с одинаковыми значенями на соответсвующих точках.}


\subsubsection{Квадрики в аффинном пространстве}

\begin{to_def} 
    Каждой квадратичной функции $Q$ на $\mathbb{A}$ ставится в соотвествие пространственная конфигурация точек $S_Q$, называемая \textit{квадрикой} (или \textit{гиперповерхностью второго порядка}) -- "геометрическое место" всех точек $\dot p \in \mathbb{A}$ $\colon Q(\dot p) = 0$. 
\end{to_def}


Любые две не гиперплоскости равны с точностью до множителя. Также понимаем, что такое центр квадрики.

\begin{to_thr} [Канонические типы квадрик]
    Случай центральной квадрикис центром симметрии в начале координат исчерпывается типами
    $$
        I_{s, r} \colon x_1^2 + \ldots  + x_2^2 - x_{s+1}^2 - \ldots  - x_r^2 = 1, 
        \hspace{0.5cm} 0 < s \leq r;
        I_{s, r}' \colon x_1^2 + \ldots  + x_2^2 - x_{s+1}^2 - \ldots  - x_r^2 = 0, 
        \hspace{0.5cm} r/2 \leq s \leq r.
    $$
    Случай нецентральной квадрики исчерпывается типами
    $$
        II_{s, r}' \colon x_1^2 + \ldots  + x_2^2 - x_{s+1}^2 - \ldots  - x_r^2 = -2 x_{r+1}, 
        \hspace{0.5cm} r/2 \leq s \leq r.
    $$

\end{to_thr}


\begin{to_def} 
    Квадрика типа $I_{n, n}$ называется \textit{эллипсоидом}, типа $I_{s, n}, s < n,$ -- \textit{гиперболидом}, типа $II_{n-1, n-1}$ -- \textit{эллиптическим парабалоидом}, типа $II_{s, n-1}$ -- \textit{гиперболическим параболоидом}. Все эти квадрики \textit{невырожденные}.

    Квадрики типа $I_{s, r}, I_{s, r}'$ при $r < n$ и типа $II_{s, r}$ при $r < n-1$ называются \textit{цилиндро}, а квадрики типа $I'_{s, n}$ -- \textit{конусами}. Это \textit{вырожденные} квадрики. 
\end{to_def}