\section{Линейные отображения}

\begin{to_def}
    Отображение $f \colon V \to W$ называется \textit{линейным}, если
    $$
        f(\vc{x} + \vc{y}) = f(\vc{x}) + f(\vc{y}), \hspace{0.5cm} f(\lambda \vc{x}) = \lambda f(\vc{x}).
    $$
\end{to_def}

С любым линейным отображением $f \colon V \to W$ ассоциируются два подпространства -- его \textit{ядро}
$$
    \Ker f = \{\vc{v} \in V \mid f(\vc{v}) = 0\}
$$
и \textit{образ}
$$
    \Im f = \{\vc{w} \in W \mid \vc{w} = f(\vc{v}) \text{ для некоторого $\vc{v} \in V$}\}.
$$

\begin{to_thr} 
    Пусть $V$ над $\mathbb{F}$, $f \colon V \to W$. Тогда $\Ker f$, $\Im f$ конечномеры и 
    $$
        \dim \Ker f +\dim \Im f = \dim V.
    $$
\end{to_thr}

\begin{proof}[$\triangle$]
    
\end{proof}