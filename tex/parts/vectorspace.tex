\section{Векторные пространства}

\begin{to_def}
	Пусть $\varkappa$ -- произвольное поле. \textbf{Векторным пространством} над $\varkappa$ называется множество $V$ элементов (векторов), удовлетворяющее следующим аксиомам:

	\noindent
	a) На $V$ бинарная операция $V \times V \rightarrow V$:
	\begin{enumerate}[label = \Roman*.]
		\item $\vc{x} + \vc{y} = \vc{y} + \vc{x}$ (коммутативность);
		\item $(\vc{x} + \vc{y}) + \vc{z} = \vc{x} + (\vc{y} + \vc{z})$ (ассоциативность);
		\item $\vc{x} + \vc{0} = \vc{x}, \, \forall \vc{x}\in V$ (нулевой вектор);
		\item $\vc{x} + (-\vc{x}) = \vc{0}, \, \forall \vc{x} \in V$ (обратный вектор) ;
	\end{enumerate}
	б) На $\varkappa \times V$ операция $(\lambda, \vc{x} \rightarrow \lambda \vc{x})$:
	\begin{enumerate}[label = \Roman*., start = 5]
		\item $1 \cdot \vc{x} = \vc{x}$ (унитарность);
		\item $(\alpha \beta) \vc{x} = \alpha  (\beta \vc{x})$ (ассоциативность);
		\item $(\alpha + \beta) \vc{x} = \alpha \vc{x} + \beta \vc{x}$;
		\item $\lambda (\vc{x} +\vc{y}) = \lambda \vc{x} + \lambda \vc{y}$.
	\end{enumerate}
\end{to_def}