\section{Векторные пространства}

\subsection{Начальные понятия}

\begin{to_def}
	Пусть $\mathbb{F}$ -- произвольное поле. \textbf{Векторным пространством} над $\mathbb{F}$ называется множество $V$ элементов (векторов), удовлетворяющее следующим аксиомам:

\begin{minipage}[t]{0.45\textwidth}
	\noindent
	a) На $V$ бинарная операция $V \times V \rightarrow V$:
	\begin{enumerate}[label = \Roman*.]
		\item $\vc{x} + \vc{y} = \vc{y} + \vc{x}$ (коммутативность);
		\item $(\vc{x} + \vc{y}) + \vc{z} = \vc{x} + (\vc{y} + \vc{z})$ (ассоциативность);
		\item $\vc{x} + \vc{0} = \vc{x}, \, \forall \vc{x}\in V$ (нулевой вектор);
		\item $\vc{x} + (-\vc{x}) = \vc{0}, \, \forall \vc{x} \in V$ (обратный вектор);
	\end{enumerate}
\end{minipage}
\hfill
\begin{minipage}[t]{0.45\textwidth}
	б) На $\mathbb{F} \times V$ операция $(\lambda, \vc{x} \rightarrow \lambda \vc{x})$:
	\begin{enumerate}[label = \Roman*., start = 5]
		\item $1 \cdot \vc{x} = \vc{x}$ (унитарность);
		\item $(\alpha \beta) \vc{x} = \alpha  (\beta \vc{x})$ (ассоциативность);
		\item $(\alpha + \beta) \vc{x} = \alpha \vc{x} + \beta \vc{x}$;
		\item $\lambda (\vc{x} +\vc{y}) = \lambda \vc{x} + \lambda \vc{y}$.
	\end{enumerate}
\end{minipage}
\end{to_def}

\begin{to_def}
	Пусть $V$ -- векторное пространство над $\mathbb{F}$, $U \subset V$ -- его подмножество, аддитивна подгруппа и переходящая в себя при умножении на скаляры. Тогда ограничение на $U$ операций в $V$ делает $U$ векторным пространством. $U$ -- \textbf{векторное подпространство} V.
\end{to_def}

\begin{to_def}
	Векторы $\vc{v}_1, \ldots, \vc{v}_n$ подпространства $V$ -- \textbf{линейно зависимы}, если $\exists$ их нетривиальная \textbf{ЛК} равная нулю. В противном случае -- линейно независимы.
\end{to_def}

\begin{to_thr}
	Если линейная система векторов линейно независима, то и всякая её подсистема также линейно независима.
\end{to_thr}	

\begin{to_thr}
	Если в $V$ $\forall \vc{e}_i \in (\vc{e}_1,\ldots,\vc{e}_s)$ -- \textbf{ЛК} векторов из $(\vc{f}_1,\ldots,\vc{f}_t)$, \textbf{то} $s \leq t$.
\end{to_thr}

\begin{to_con}
	$\forall$две эквивалентные ЛНеЗ системы векторов в $V$ содержат одинаковое число векторов.
\end{to_con}

\subsection{Размерность и базис}
\begin{to_def}
	\textbf{Ранг} системы векторов -- число векторов в любой max ЛНеЗ подсистеме.
\end{to_def}

\begin{to_def}
	$V$, содержащее $n$ ЛНеЗ векторов, в котором не ЛНеЗ систем большего ранга, называется \textbf{n-мерным}. $\dim_\mathbb{F} V = n$. 
\end{to_def}

\begin{to_def}
	$\dim_\mathbb{F} V = n$. Любая система и $n$ независимых векторов называется \textbf{базисом} пространства $V$.
\end{to_def}

\begin{to_thr}
	$\dim_\mathbb{F} V = n$ c $( \vc{e}_1,\ldots,\vc{e}_n)$. Тогда: 1) $\forall \vc{v} \in V$ $\exists!$ ЛК из векторов базиса; 2) любую систем из $s<n$ ЛНеЗ векторов можно дополнить до базиса. 
\end{to_thr}

\begin{to_def}
	$\dim_\mathbb{F} V = n$ c $( \vc{e}_1,\ldots,\vc{e}_n)$. $\lambda_i \in \mathbb{F}$ называются \textbf{координатами вектора}: $\vc{v} = \lambda_1 \vc{e}_1 + \ldots + \lambda \vc{e}_n$. 
\end{to_def}

\begin{to_thr}
	При переходе $(\vc{e}_1,\ldots,\vc{e}_n) \leadsto ( \vc{e'}_1,\ldots,\vc{e'}_1)$, определяемом $A \in \mathcal{M}_{n n}$, координаты $\vc{v}$: $\lambda_j^\text{новые}$ выражаются через $\lambda_i^\text{старые}$ при помощи обратимого линейного преобразования с $A^{-1}$.
	\label{transition_matrix}
\end{to_thr}

\begin{to_def}
	$V$ и $W$ над $\mathbb{F}$ -- \textbf{изоморфны}, если  $\exists$ биективное $f \colon V \to W:$ $f(\alpha \vc{u} + \beta \vc{v}) = \alpha f(\vc{u}) + \beta f(\vc{v})$.
\end{to_def}

\begin{to_thr}
	Все $V$ одинаковой $\dim = n$ над $\mathbb{F}$ изоморфны (координатному пространству $\mathbb{F}^n$).
\end{to_thr}

\begin{to_thr}
	$U, W$ -- конечномерные подпространства $V$. Тогда : $\dim(U + W) = \dim U + \dim W - \dim(U \cap W)$.
\end{to_thr}

\begin{to_def}
	Если $\forall \vc{u} \in U$ может быть однозначно представлен в виде $\vc{u} = \vc{u}_1 + \ldots + \vc{u}_m$. То сумма называется \textbf{прямой}: $U = U_1 \oplus \ldots \oplus U_m$.
\end{to_def}

\begin{to_thr}
	$U = U_1 \oplus \ldots \oplus U_m$ -- прямая $\Longleftrightarrow$ $U_i \cap (U_1 + \ldots + U_m) = 0$, для  $i=1, \ldots, m$.
\end{to_thr}

\begin{to_thr}
	$U = U_1 \oplus \ldots \oplus U_m$ -- прямая $\Longleftrightarrow$ $\dim U = \sum\limits_{ i=1 }^{ m } \dim U_i$.
\end{to_thr}

\begin{to_thr}
	$\forall m$-мерного $U \subset V$ ($\dim V = n$) $\exists W (\dim W = n-m):$ $V = U \oplus W$.
\end{to_thr}